\documentclass[a4paper, 12pt, titlepage]{extarticle}
  \usepackage[bib,math]{mypreamble} % моя преамбула
  \usepackage[babel,protrusion=true,expansion]{microtype}

  \newcommand{\thetitle}{Каталог прочитанной литературы к кандидатской}
  \newcommand{\theauthor}{Иван Новиков}

  \author{\theauthor,\\ кафедра физики и информационных систем КубГУ,\\ направление аспирантуры 01.04.10}
  \title{\thetitle}
  \date{\today{} \currenttime}

  \SetPdfTitleAndAuthor{\thetitle}{\theauthor}

  % ----- дополнительные команды и настройки -----

  % inline библиографические ссылки
  \usepackage{bibentry}
  \nobibliography*

  % Убираем отступ первого абзаца, чтобы не тратить место для bibentry
  \setlength{\parindent}{0pt}

  \newenvironment{catentry}[2]%
    {\subsubsection{#2}\bibentry{#1}\begin{description}}%
    {\end{description}}
  \newcommand{\itabstract}[1]{\item[Аннотация] {\footnotesize #1} }
  \newcommand{\itacquaint}[1]{\item[Ознакомление] #1}
  \newcommand{\ituseful}[1]{\item[Полезно] #1}
  \newcommand{\itdate}[3]{\item[Дата] \formatdate{#1}{#2}{#3}}
  \newcommand{\iturl}[1]{\item[URL] \url{#1}}
  \newcommand{\itdoi}[1]{\item[DOI] \url{http://dx.doi.org/#1}}

\begin{document}

%----------------------- титульный лист ------------------------

  \maketitle

%------------------------- содержание -------------------------

  \microtypesetup{protrusion=false} % отключить protrusion для содержания
    \clearpage
    \tableofcontents
  \microtypesetup{protrusion=true}

%-------------------------- введение --------------------------

  \sectiontoc{Введение}\label{sec:intro}

  Каждый прочитанный источник попадает в одну из категорий~--- книги, статьи, конференции, интернет.
  Для каждого указывается библиографическая запись, оригинальный \eng{abstract} (аннотация),
  переведённый на русский, чем полезен, степень ознакомления и дата последнего изменения, а также
  URL или DOI (если есть).

  Формат в данный момент не окончен, в дальнейшем нужно:
  \begin{enumerate}
    \item \todo{Сделать формат и макросы более гибкими, но строгими}
    \item \todo{Визуализация степени ознакомления}
    \item \todo{Сделать алфавитный указатель для быстрого поиска и правильную сортировку}
    \item \todo{Разобраться с корректной работой bibentry: пока почему-то не может работать с
      \textbackslash nobibliography, только с \textbackslash bibliography}
    \item \dots
  \end{enumerate}

  \section{Книги}

  \subsection{Математика}

  \begin{catentry}{rashevsky-riemann}{Риманова геометрия и тензорный анализ~--- Рашевский}
    \itabstract{
      В настоящей монографии в развернутом изложении и со всесторонним освещением
      предмета автором представлен материал, включающий самое основное и важнейшее в области
      тензорного анализа и римановой геометрии. %\par
      Отличительной чертой книги являются выходы из области чистого тензорного анализа и римановой
      геометрии в механику и физику (особое внимание в этом плане уделено теории относительности).
      Рассматриваются псевдоевклидовы и псевдоримановы пространства, пространства афинной связности.
      На ряде примеров даны основные идеи теории геометрических объектов, в том числе теория
      спиноров в четырехмерном пространстве. Изложение дополнено также рядом частных вопросов
      фундаментального значения (теория кривых и гиперповерхностей в римановом пространстве и др.). %\par
      Книга предназначена специалистам в области тензорного анализа и римановой геометрии,
      инженерам, может также служить учебником для студентов вузов.
    }
    \itacquaint{
      Прочитал почти полностью
    }
    \ituseful{
      Определение \term{пространства афинной связности}, понятие о \term{коэффициентах
      связности} $\gamma^j_{kl}$ и \term{тензоре кривизны} $R^{...n}_{klm.}$ и математический вывод
      всех этих вещей. Данные математические конструкции лежат в основе формализма многих
      \term{калибровочных теорий дефектов} в \term{сплошных средах}, да даже и некоторых
      классических \term{теорий дислокаций}.
    }
    \itdate{30}{4}{2015}.
    \iturl{http://www.ozon.ru/context/detail/id/4787539}
    \iturl{http://pskgu.ru/ebooks/rashewsky.html}
  \end{catentry}

  \subsection{Физика}

  \begin{catentry}{matare-defects}{Электроника дефектов в полупроводниках~--- Матаре}
    \itabstract{
      Первая в мировой литературе монография, специально посвященная влиянию
      различных отклонений от периодической структуры кристаллов (дислокации,
      границ зерен и т. д.) на неравновесные электронные процессы в полупроводниках.
      Эта проблема приобрела в последние годы особую актуальность в связи с
      использованием в полупроводниковой технике тонких пленок, сложных структур
      и интегральных схем. В основу книги легли лекции, читавшиеся автором в
      ведущих технических университетах США; текст содержит большое количество
      иллюстративного, в частности графического, материала, что делает книгу
      полезной для практических целей. %\par
      Книга представляет интерес для специалистов в области физики твердого тела,
      физической электроники и электронной техники, а также для аспирантов и
      студентов университетов, физико-технических и радиоэлектронных вузов.
    }
    \itacquaint{
      Начал читать (1 и 2 главы)
    }
    \ituseful{
      Взгляд на \term{дефекты} с точки зрения \term{классической} механики~--- он устаревший и
      неактуальный, но знать основы нужно. Хорошая (на момент 70-х) \term{эмпирическая} база.
    }
    \itdate{1}{12}{2014}
  \end{catentry}

  \section{Статьи}

  \subsection{Классические подходы}

  \begin{catentry}{hochrainer-cdd}{Непрерывная динамика дислокаций~--- Хокрайнер}
    \itabstract{
      Пластическая деформация металлов~--- это результат движения и взаимодействия дислокаций~---
      линейных дефектов кристаллической структуры. Непрерывные модели пластичности, однако, остаются
      на сегодняшний день преимущественно феноменологическими, не рассматривают движение дислокаций
      и терпят неудачу когда поведение материала приобретает масштабную зависимость. В этой работе
      мы представляем новую теорию пластичности, основанную на систематическом физическом усреднении
      кинематики и динамики дислокационных систем. Мы показываем, что эта теория способна
      предсказывать эволюцию микроструктур и масштабные эффекты в соответствии с экспериментами и
      дискретным моделированием дислокаций. Теория основывается на только лишь четырёх внутренних
      переменных на каждую систему скольжения и включает физические граничные условия, скопления
      дислокаций, кривизну дислокаций, размножение дислокаций и потерю дислокаций. Поэтому предлагаемая
      теория является большим шагом на пути к физической теории пластичности кристаллов.
    }
    \itacquaint{
      Прочитал, перевёл частично на русский
    }
    \ituseful{
      Подход, хотя и классический, но активно пользующийся понятиями \term{тензора плотности
      дислокаций} $\alpha_{ij}$, а также некоторых важных скалярных величин. Кроме того, описывает
      \term{численное моделирование} на основе этой теории.
    }
    \itdate{23}{4}{2015}
    \itdoi{10.1016/j.jmps.2013.09.012}
  \end{catentry}

  \subsection{Калибровочные и геометрические подходы}

  \begin{catentry}{sahoo-gauge-I}{Калибровочные теории дефектов в упругих средах~--- Валсакумар}
    \itabstract{
      Мы предлагаем выборочный обзор калибровочных теорий дефектов в упругой сплошной среде. После
      введения основных геометрических понятий механики сплошных сред при наличии дефектов, вводятся
      классические уравнения динамики дефектов, включающие тензоры плотности дислокаций и дисклинаций.
      Вкратце обсуждается математическая структура калибровочных теорий. Мы качественным образом
      касаемся некоторых типичных современных работ, охватывающих калибровочные теории Янга-Миллса и
      калибровочные теории гравитационного типа.
    }
    \itacquaint{
      Прочитал почти полностью (1--4 раздел)
    }
    \ituseful{
      Очень информативное (в плане истории) введение. Последовательно даются базовые формулы
      связности, причём не только в терминах $\Gamma^j_{kl}$, но и \term{1-формы связности}
      $\omega^i_k$ и \term{внешнего произведения} $\wedge$ (раздел 2). Описывается связь
      \term{тензора кручения} $T^{i}_{jk}$ и \term{тензора кривизны} $R_{klmn}$ с \term{тензором
      плотности дислокаций} $\alpha_{ij}$ и \term{тензором несовместности} $\eta_{ij}$,
      соответственно (включая кто первым её предложил), с упоминанием и \term{тензора деформации}
      $\beta^i_k$ (раздел 3). Сложные термины \term{расслоения} и действия в них \term{групп Ли}
      объяснены кратко и довольно простыми словами (раздел 4).
    }
    \itdate{1}{5}{2015}
  \end{catentry}

  \begin{catentry}{hasebe-ftmp}{Полевая теория многомасштабной пластичности~--- Хасебе}
    \itabstract{
      Предпринимается попытка воспроизвести экспериментально наблюдаемые неоднородные деформационные
      структуры на основе ПТМП (полевой теории многомасштабной пластичности), в то время как ПЭМ для
      \todo{sheared} образцов монокристаллов четырёх типичных кристаллографических ориентаций и
      результаты метода ДОЭ-Вилкинсона для поликристаллов выбраны в качестве современных успешных
      примеров, с целью в конечном итоге разработать новую технологию для расчёта \todo{in situ/ex situ
      local-global} неоднородностей во время упруго-пластической деформации совместно с этими
      экспериментальными технологиями. Проведено МКЭ-моделирование  пластичности с использованием
      модели несовместности, основанной на ПТМП в связи с рабочей гипотезой, названной
      \todo{flow-evolutionary} законом, проявлением которого является соотношение между тензором
      несовместности и тензором энергии-импульса (диаграмма двойственности). Показано не только
      успешное воспроизведение зависящих от ориентации дислокационных субструктур и развивающихся
      внутри зёрен деформационных структур, но также и связанный поток энергии с развившимися
      неоднородностями, визуализированными на соответствующей диаграмме двойственности.
    }
    \itacquaint{
      Прочитал, перевёл частично на русский
    }
    \ituseful{
      Неплохое введение в применение \term{неримановой геометрии} к описанию \term{дефектов}.
      Намечены способы её объединения с \term{калибровочной теорией поля} и \term{квантовой теорией
      поля}. Упоминается связь \term{тензора кручения} $S^{..j}_{kl.}$ и  \term{тензора кривизны}
      $R^{...n}_{klm.}$ с \term{тензорами деформации} $\beta_{ij}$ и $\varepsilon_{ij}$, а также с
      \term{тензором плотности дислокаций} $\alpha_{ij}$ и \term{тензором несовместности}
      $\eta_{ij}$, соответственно.
      Подробно описано \term{численное моделирование} на основе этой теории.
    }
    \itdate{1}{5}{2015}
    \itdoi{10.2320/matertrans.M2013226}
  \end{catentry}

  \begin{catentry}{liu-nematic}{Калибровочная теория дислокаций в 2D-нематиках~--- Заанен}
    \itabstract{
    }
    \itacquaint{
    }
    \ituseful{
    }
    \itdate{2}{5}{2015}
    \itdoi{10.1103/PhysRevB.91.075103}
  \end{catentry}

  %\begin{catentry}{bibtex-key}{Статья (кратко)~---автор}
  %  \itabstract{
  %  }
  %  \itacquaint{
  %  }
  %  \ituseful{
  %  }
  %  \itdate{1}{5}{2015}
  %  \itdoi{doi}
  %\end{catentry}

  \PrintBibliography

\end{document}
