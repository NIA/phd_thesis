\documentclass[a4paper, 14pt, titlepage]{extarticle}
  \usepackage[bib,figtable,newpage,titlegost]{mypreamble} % моя преамбула
  \usepackage[babel,protrusion=true,expansion]{microtype}
  \usepackage{setspace} % для \begin{spacing}

  \newcommand{\thetitle}{Философские, историко-научные и теоретико-методологические основания
  моделирования дефектов в кристаллических веществах}
  \newcommand{\theauthor}{Иван Новиков}

  \author{\theauthor,\\ кафедра физики и информационных систем КубГУ}
  \title{\thetitle}
  \date{\today{} \currenttime}

  \SetPdfTitleAndAuthor{\thetitle}{\theauthor}

\begin{document}

%----------------------- титульный лист ------------------------

  \thispagestyle{empty}
  \begin{center}
  \begin{spacing}{1.0}
  {\small МИНИСТЕРСТВО ОБРАЗОВАНИЯ И НАУКИ РОССИЙСКОЙ ФЕДЕРАЦИИ\\
  \textit{Федеральное государственное бюджетное образовательное учреждение\\
  высшего профессионального образования}}\\
  \textbf{\large«Кубанский государственный университет»\\
  (ФГБОУ ВПО «КубГУ»)}

  \vspace {5mm}

  Кафедра философии

  \vspace {4cm}

  \textbf{\large Р~Е~Ф~Е~Р~А~Т}\\
  по дисциплине <<Логика и методология научного познания>>

  \vspace {0.5cm}

  { \scshape \thetitle }

  \vspace {1.5cm}

  \hfill\begin{minipage}{0.63\textwidth}
    Выполнил:\\
    аспирант кафедры физики и информационных систем
    физико-технического факультета\\
    НОВИКОВ Иван Александрович
    %Направление аспирантуры 01.04.10 Физика полупроводников
    \vspace{1cm}

    Научный руководитель:\\
    д-р физ.-мат. наук, профессор,\\
    Богатов Н.\,М.
  \end{minipage}

  \vfill

  Краснодар \the\year
  \end{spacing}
  \end{center}

%------------------------- содержание -------------------------

  \microtypesetup{protrusion=false} % отключить protrusion для содержания
    \clearpage
    \tableofcontents
  \microtypesetup{protrusion=true}

%-------------------------- введение --------------------------

  \sectiontoc{Введение}\label{sec:intro}

  Полупроводники в настоящее время имеют огромное значение, поскольку лежат в основе всей
  современной электроники и компьютерной техники, а также применяются в солнечной энергетике и
  оптоэлектронике. При этом изучение структурных дефектов в полупроводниках представляет особый
  интерес, с~одной стороны, поскольку они могут существенно влиять на электрические свойства
  полупроводниковых приборов (как отрицательно, так и положительно), а с другой~--- по причине их
  участия в механических процессах, таких как пластичность, причём последнее имеет место в широком
  смысле: не только для полупроводников, но и для других кристаллических веществ.

  Эта область представляет чрезвычайный интерес для учёного, стремящегося к познанию наиболее общих,
  фундаментальных закономерностей природы. В физике полупроводников, особенно в теории структурных
  дефектов, переплетаются такие глубочайшие современные физические и математические парадигмы, как квантовая
  электродинамика, калибровочная теория поля, неевклидова и нериманова дифференциальная геометрия и
  теория групп Ли. И это неудивительно, ведь уже в том факте, что микроскопические дефекты,
  затрагивающие считанное число атомов, ярко проявляются на макро- и мега-уровне, содержится
  глубокое диалектическое противоречие, которое может быть снято только в конкретно-всеобщей теории.
  Построение такой теории в данный момент является одной из важнейших задач современной физики.

  Приведём определения основных терминов, возникающих при рассмотрении темы структурных дефектов в
  полупроводниках: полупроводник, его структура, дефекты в этой структуре и, наконец, дислокации как
  наиболее важный вид дефектов.

  \paragraph{Полупроводник.} Термин \term{полупроводник} возникает при рассмотрении такого важного
  свойства материи, как способность проводить электрический ток. По отношению к этому свойству все
  вещества можно разделить на три класса: металлы, полупроводники и диэлектрики. Выбор критерия
  такой классификации не является очевидным. Казалось~бы, проще всего использовать в качестве такого
  критерия удельное сопротивление вещества.

  \begin{mytable}{Диапазоны удельного сопротивления $\rho$ классов веществ}{tab:rho}
    \begin{tabular}{r|l}
      Класс вещества & $\rho$, Ом$\cdot$см \\
      \hline
      Проводники     & $10^{-6}$ .. $10^{-4}$\\
      Полупроводники & $10^{-4}$ .. $10^{10}$\\
      Диэлектрики    & $> 10^{10}$\\
    \end{tabular}
  \end{mytable}

  Однако, как следует из таблицы \ref{tab:rho}, диапазоны удельного сопротивления различных классов
  перекрываются. Например, сернистый кадмий (полупроводник) в~зависимости от технологии изготовления
  может иметь удельное сопротивление как ниже, чем нихром (металлический сплав), так и выше, чем
  стекло (диэлектрик). Следовательно, одно только значение удельного сопротивления не может служить
  однозначным критерием.

  Решение этого вопроса можно найти, если проследить, как изменяется удельное сопротивление при
  понижении температуры. Сопротивление металлов при этом уменьшается: в~них свободные электроны
  присутствуют независимо от температуры, а подвижность увеличивается за счёт ослабления тепловых
  колебаний решётки. У полупроводников же, напротив, удельное сопротивление повышается при снижении
  температуры: свободные носители заряда возникают в нём за счёт теплового возбуждения
  \term{(равновесные носители)}, а~также при освещении и ядерном облучении \term{(неравновесные)}.
  Этот процесс образования носителей (как равновесных, так и неравновесных) сильно зависит от
  структуры полупроводникового материала и от наличия примесей.

  С~учётом вышесказанного, можно дать следующее определение. \term{Полупроводник}~--- это такой кристаллический
  материал, который способен проводить электрический ток, но гораздо хуже большинства металлов, а
  именно: при комнатной температуре имеет удельное сопротивление от $10^{-6}$ до $10^{-4}$
  Ом$\cdot$см, которое сильно зависит от
  \begin{enumerate*}[label=\asbuk*)]
    \item структуры вещества
    \item вида и количества примесей,
    \item внешних условий: температуры, освещения, облучения ядерными частицами, электромагнитного
      поля~\cite{shalimova-semiconductors}.
  \end{enumerate*}

  \paragraph{Структура полупроводника.} Физические свойства полупроводников определяются, главным
  образом, тем, что они являются кристаллами. С этим связаны два основных понятия.

  \term{Кристаллическая решётка}~--- присущее кристаллам регулярное расположение частиц (атомов, их
  ядер, ионов, молекул, электронов), характеризующееся периодичной повторяемостью в трёх
  измерениях~\cite{physenc-crystall}; вспомогательный геометрический образ, вводимый для анализа
  строения кристалла.

  \term{Кристаллическая структура}~--- такая совокупность атомов, в которой с каждой точкой
  кристаллической решётки связана определённая группа атомов, называемая \term{мотивной единицей}. Можно считать,
  что структура возникает в результате синтеза решётки и мотивной единицы, в результате размножения
  мотивной единицы группой трансляции \cite{wiki-cryst-struct}.
  Обычно, говоря о кристаллической структуре, подразумевают среднее во времени расположение атомных
  ядер (так называемая статическая модель); более полная информация включает сведения об амплитудах и частотах
  колебаний атомов (динамическая модель), а также о распределении электронной плотности в межъядерном
  пространстве \cite{chemenc-crystall}.

  \paragraph{Структурные дефекты.} \term{Дефекты} в кристаллах~--- устойчивые нарушения правильного
  расположения атомов или ионов в узлах кристаллической решётки, соответствующего минимуму потенциальной
  энергии кристалла. Иначе говоря, структурные дефекты, или несовершенства~--- это отклонения
  расположения атомов от идеальной кристаллической структуры. Существуют большое количество
  различных видов дефектов, их можно классифицировать по размерности.
  \begin{enumerate}
    \item Нулевой размерности~--- точечные дефекты, например, вакансии и междоузельные атомы.
    \item Одномерные~--- линейные дефекты~--- дислокации.
    \item Двумерные~--- поверхностные дефекты, например, границы зёрен.
    \item Трёхмерные~--- объёмные дефекты, например, поры \cite{matare-defects}.
  \end{enumerate}

  \paragraph{Дислокации}\unskip--- дефекты кристаллической решётки, искажающие правильное расположение
  атомных (кристаллографических) плоскостей. Два базовых типа: краевая и винтовая дислокации.
  \term{Краевая дислокация} соответствует атомной плоскости, обрывающейся внутри кристалла (рис.~\ref{fig:dislocation}).
  \term{Винтовую дислокацию} можно представить себе как результат сдвига на период решётки одной
  части кристалла относительно другой вдоль некоторой полуплоскости параллельно её краю, играющему
  роль оси дислокации. % --- Физическая энциклопедия (http://www.femto.com.ua/articles/part_1/1036.html)

  \includefigure[width=0.2\textwidth]{Краевая дислокация в кубической решётке}{dislocation}
  % TODO рисунок с винтовой дислокацией

  Сказанное выше заключает в себе всё, что необходимо знать для того, чтобы иметь представление о
  том, какие физические законы лежат в основе физики дефектов в кристаллах. Однако понятие
  структурного дефекта тем самым ещё не раскрыто полностью, поскольку пока были перечислены только
  лишь некоторые его особенные моменты.

  \section{Понятие дефекта в философии физико-химических наук}

  Прежде чем подойти напрямую к сложному понятию дефекта, рассмотрим те основные, базовые категории,
  возникающие в философии физики, которые наиболее применимы к вопросу о дефектах структуры вещества.

  \subsection{Базовые категории философии физики}

  Категории материи, пространства, симметрии и энергии являются базовыми для философии физики в
  целом. Для физики же кристаллических дефектов они приобретают особенное значение, поскольку
  кристаллическое вещество~--- это наиболее симметрично организованная в пространстве форма материи,
  при которой атомы занимают строго упорядоченные, симметричные позиции, соответствующие минимуму потенциальной
  энергии электромагнитного взаимодействия между ними. Рассмотрим эти категории по очереди.

  \subsubsection{Материя}

  Материя является одной из наиболее многозначных философских категорий, определение которой сильно
  разнится в зависимости от философской традиции (например, оно почти противоположно в материализме
  и идеализме). В любом случае эта категория используется для обозначения физической субстанции
  вообще, в противоположность сознанию или духу. Вопрос отношения между материей и сознанием (что
  первично?) в некоторых философских концепциях, например в марксизме, даже называется <<основным
  вопросом философии>>. Однако же в философии физико-химических наук он уже не имеет такого
  первостепенного значения, поскольку частные физические науки, как правило, изучают материю как
  таковую, её внутренние свойства и законы вне её связи с сознанием.

  Соответственно, в философии физики \term{материя}~--- это основной предмет физики, та
  действительность, определяющими характеристиками которой являются протяжённость, место в
  пространстве, масса, вес, движение, инерция, сопротивление, непроницаемость, притяжение и
  отталкивание, или какая-то комбинация этих свойств; это внешняя причина чувственного
  опыта~\cite{enc-phylosophy-04}. Основными формами материи, выделяемыми в физике в настоящий момент, являются:
  \begin{itemize}
    \item \term{вещество}~--- элементарные частицы и состоящие из них атомы, молекулы и
      макроскопические тела;
    \item \term{поле}~--- фундаментальные взаимодействия: электромагнитное, сильное,
      слабое и гравитационное;
    \item \term{тёмная материя и тёмная энергия}, структура и происхождение которых пока не изучены.
  \end{itemize}

  В физике дефектов полупроводников изучаются как вещество, составляющее кристалл, так и
  периодические поля, действующие в кристаллической решётке.

  \subsubsection{Пространство}

  Выше было указано, что протяжённость, как способ отношения к пространству, является одной из
  основных характеристик материи. Когда мы переходим к философии физики, то категория пространства,
  как и категория материи, оказывается достаточно многозначной.

  В общефилософском смысле \term{пространство}~--- фундаментальное (наряду со временем) понятие
  человеческого мышления, отображающее множественный характер существования мира, его
  неоднородность. Множество предметов, объектов, данных в человеческом восприятии одновременно,
  формирует сложный пространственный образ мира, являющийся необходимым условием ориентации любой
  человеческой деятельности \cite{enc-phylosophy-04}.

  В физике же чётко выделяются два связанных, но раздельных смысла термина <<пространство>>.
  В классической (в частности, <<школьной>>) физике так называемое \term{обычное пространство}, или же иначе
  \term{физическое пространство}~--- это трёхмерное пространство нашего повседневного мира, в котором
  определяется положение физических тел, в котором происходит механическое движение, геометрическое
  перемещение различных физических тел и объектов. При этом современная физика идёт далеко вперёд
  в своём определении пространства, заимствуя из математики вдобавок к вышесказанному понятие \term{абстрактного
  пространства}~--- математического объекта, базирующегося на теории множеств и математической
  топологии. Математический аппарат теории относительности, квантовой физики и теории дефектов в
  кристаллах широко использует различные по структуре \term{векторные (линейные) пространства},
  которые могут быть как представлением обычного четырёхмерного пространства-времени (в простом
  случае), так и совершенно не связанной с ним теоретической моделью, параметрами которой вместо
  протяжённости вдоль разных координатных осей уже являются так называемые \term{обобщённые
  координаты}, представляющие степени свободы системы.

  В современных теориях физики дефектов широко применяется именно второе, сложное понимание пространства.

  \subsubsection{Симметрия}

  Выше было сказано, что по отношению к пространству вещество (и поле) в кристаллах упорядочены
  периодически, или иначе говоря, симметрично. \term{Симметрия}~--- (от греч. <<соразмерность>>),
  понятие, характеризующее переход объектов в самих себя или друг в друга при осуществлении над ними
  определённых преобразований, а в широком смысле~--- свойство неизменности (инвариантности)
  некоторых сторон, процессов и отношений объектов относительно некоторых преобразований~\cite{enc-phylosophy-83}.

  В физике именно это, широкое понимание приобретает особенную важность, в~том числе в свете
  \term{теоремы Нетёр}, утверждающей, что каждой непрерывной симметрии соответствует некоторый закон
  сохранения~\cite{noether-invariante}. В частности,
  \begin{itemize}
    \item однородности времени соответствует закон сохранения энергии,
    \item однородности пространства соответствует закон сохранения импульса,
    \item изотропии пространства соответствует закон сохранения момента импульса,
    \item калибровочной симметрии соответствует закон сохранения электрического заряда и т. д.
  \end{itemize}

  Однако же в случае кристаллов имеет место не непрерывная, а дискретная симметрия: кристалл
  переходит в себя не при произвольном смещении или повороте, а только при кратном заданному.
  Иначе говоря, имеет место более высокая степень упорядоченности, чем в веществе с непрерывной
  симметрией. Это и обуславливает их важные свойства, такие как анизотропия и полупроводниковые
  свойства. Именно поэтому отклонения от этой симметрии могут резко изменять свойства
  кристаллического вещества. Тем самым мы вплотную подошли к понятию дефекта в философии физики.

  % TODO \subsubsection{Энергия} - если хватит времени

  \subsection{Содержание понятия дефекта}

  Чтобы полностью раскрыть понятие дефекта, нужно рассмотреть его на уровне всеобщего, особенного и
  единичного. С точки зрения всеобщей логики необходимо исходить из принципа отрицательности,
  который присущ как материи, так и мысли. Дефект в~широком смысле~--- это отклонение от нормы или
  должного, то есть отрицательность собственной сущности вещей. С точки зрения терминологии
  диалектики можно сказать, что дефект~---это несоответствие того, чем нечто является в себе, тому,
  что оно есть для себя~\cite{hegel-logic}.

  % TODO HIGH LEVEL PHILOSOPHY HARDCORE - что-то ещё про отрицательность.

  В отношении физики кристаллических тел, дефект~--- это отрицательность идеальной структуры,
  несоответствие наличного бытия атомов, их реального расположения в пространстве той симметричной
  структуре, которой должен обладать данный материал в данных условиях согласно идеальной модели кристаллической решётки.

  Существует большое количество особенных форм дефектов в кристаллической решётке, основные из
  них были перечислены выше во введении: точечные дефекты, дислокации, границы зёрен, поры и другие.
  Однако не все виды дефектов в равной степени оказывают влияние на свойства материала. В первую очередь
  рассматриваются дислокации, поскольку они, будучи протяжёнными дефектами, наиболее сильно проявляют себя на макроуровне. Накоплен большой
  опыт наблюдения отдельных дислокаций и их скоплений с помощью таких методов, как просвечивающая
  электронная микроскопия и дифракция отражённых электронов~\cite{hasebe-ftmp}. Показано также, что
  именно дислокации играют важную роль в механизме пластичности (неупругой, остаточной деформации).

  Для перехода от особенного к единичному, или конкретно-всеобщему, все эти особенные формы должны
  быть подвергнуты снятию. На данный момент в том и заключается основной интерес к данной проблемы:
  такой конкретно-всеобщей теории дефектов кристаллических тел, которая могла бы описать все эти
  виды дефектов, выделив их общую логическую и физическую сущность, пока не существует. Впрочем,
  отдельные учёные в этом стремлении достигли определённого успеха, о чём сказано в следующем разделе.

  \section{История и актуальность проблемы моделей дефектов}

  Раскрыв понятие дефекта в физике, приведём краткий историко"=научный обзор такой области физики,
  как физика дефектов в кристаллических веществах. По его итогам в данном разделе сформулирована
  основная проблематика, лежащая в основе планируемого диссертационного исследования.

  \subsection{Развитие подходов к изучению дислокаций}

  Поскольку выше было показано, что среди различных видов дефектов в кристаллах наибольший интерес
  (особенно в случае полупроводников) представляют дислокации, то далее описана история изучения именно этих дефектов.
  %
  С начала ХХ~в. дислокации изучались вне физики полупроводников как свойства произвольных
  кристаллов. Механическая теория дислокаций развивалась как раздел металловедения, благодаря ей
  удалось объяснить механизм пластических деформаций. После того, как началось изучение
  полупроводников для использования их в электронных приборах, были также изучены электрические
  свойства дислокаций. Наконец, уже в наши дни предпринято множество попыток и проделан большой
  объём работы по созданию общей теории дислокаций и пластичности, как в России, так и за рубежом.
  Причём особый интерес вызывают так называемые \term{калибровочные теории}, использующие мощный
  аппарат неевклидовой дифференциальной геометрии и понятие инвариантности относительно некоторой
  топологической группы локальных преобразований. Рассмотрим эти этапы чуть подробнее.

  Вольтерра впервые привёл идею и классификацию линейных дефектов в кристаллах и проделал расчёт
  полей упругих напряжений вокруг них ещё в 1907~г.~\cite{volterra-1907}, при этом сам термин
  \term{дислокация} появился позже: он был предложен в 1934~г. Г.~Тейлором, который и описал связь
  дислокаций с явлением пластичности металлов~\cite{taylor-plastic}.
  %
  Изначально механика дислокаций изучалась не количественно, а качественно, при этом сами дислокации
  пока оставались чисто теоретической концепцией, наблюдать которую техника того времени ещё не
  позволяла. Лишь в 1956 году с помощью просвечивающего электронного микроскопа было произведено
  первое наблюдение дислокаций. Стали строиться первые теории дефектов, которые, как правило,
  основывались на классической теории упругости, хотя энергия дефектов и энергия взаимодействия
  отдельных узлов решётки, разумеется, рассчитывалась исходя из квантово"=механической модели
  энергетических уровней атома. Ретроспективный взгляд на этот процесс приведён в статье
  Дж.~Хирта~\cite{hirth-history}.  С 50-х по 70-е годы XX~в. наблюдается большой всплеск научной
  активности по вопросу о дефектах в кристаллах и их электрических свойствах, главным образом в
  связи с возникновением и развитием полупроводниковой электроники, на функционирование которой
  дефекты кристаллической структуры полупроводника оказывают решающее влияние. Накопленная благодаря
  этому богатейшая эмпирическая база объединена и обобщена в классической монографии
  Г.~Матаре~\cite{matare-defects}, в которой он, кроме того, приводит обзор существующих на тот
  момент теоретических и феноменологических моделей.

  Здесь следует сделать небольшое отступление и обратить внимание на процессы, которые в то время
  происходили в других областях физики, на первый взгляд не связанных с данной,~--- физике
  элементарных частиц и гравитации. Начиная ещё с 1940-х годов предпринимались попытки перенести
  успехи классической теория поля, восходящей ещё к электродинамике Максвелла (1864), чтобы
  объединить общую теорию относительности Эйнштейна, квантовую механику и электродинамику. Ключевой
  особенностью Максвелловской теории поля при этом оказывается её свойство симметрии относительно
  определённых преобразований. Если для электродинамики такие преобразования называются
  преобразованиями Лоренца, то в общем случае такое свойство получило название \term{калибровочной
  инвариантности} (англ. \eng{gauge invariance}), а теории, обладающие таким свойством~---
  \term{калибровочными теориями} (англ. \eng{gauge theory}). Уже 1980-м годам калибровочные теории
  имели общепризнанный успех в теории элементарных частиц, гравитации и ядерных взаимодействий,
  поскольку позволили объяснить многочисленные явления, не вписывающиеся в прежние модели.

  Этот успех калибровочных теорий поля вдохновил многих исследователей применить калибровочный
  подход в теориях поля, применявшихся для описания дефектов в кристаллических веществах. Однако
  к~этой области применение калибровочных теорий проходило не так просто и не с~такими впечатляющими
  успехами, что вызвало активную дискуссию по поводу того, насколько такое применение уместно и
  необходимо. В 1982 году Кронер подвёл итоги этой дискуссии словами о том, что <<среди учёных есть
  как оптимизм, так и пессимизм по этому вопросу, но почти все сошлись во мнении, что исследования в
  этом направлении необходимо продолжать>>~\cite{kroner-gauge}.

  В обзорной статье Д.~Саху и М.~Валсакумара приводится история и классификация калибровочных теорий
  дефектов в упругих твёрдых кристаллических телах~\cite{sahoo-gauge-I,sahoo-gauge-II}. Точное
  определение того, какую теорию считать калибровочной, по их мнению, дать сложно, при этом распространёнными характеристиками
  таких теорий являются использование таких понятий дифференциальной геометрии, как метрика,
  связность, тензоры кручения и кривизны, а также постулат о том, что Лагранжиан системы зависит от
  этих математических величин. Авторы этого обзора разделяют все калибровочные теории на два типа:
  первый~--- теории Янга-Миллса, похожие на теорию электрослабого взаимодействия, второй~--- теории
  гравитационного типа, более близкие по математическому аппарату к общей теории относительности.
  К настоящему дню возникло огромное количество идей как в первом, так и втором русле калибровочных
  теорий, так что за всеми существующими подходами даже сложно уследить. Авторы обзора заключают:
    % There is as yet no fundamental theory of
    % defects in solids. The classical Kossecka-dewit equations of defect dynamics offers some guideline.
    % However, even these equations do not contain the signature of point defects. An ideal gauge theory
    % of defects should be consistent with these classical equations in appropriate limit. At the present
    % time, it is perhaps fair to say that much more investigation need to be carried out in this area. Also,
    % urgent attention need to be paid to obtain gauge theory solutions of specific defects.
  <<Пока не существует фундаментальной теории дефектов в твёрдых телах. Идеальная калибровочная
  теория дефектов в твёрдых средах должна согласовываться с классическими уравнениями в определённых
  пределах. В настоящее время будет справедливо сказать, что ещё очень много исследований должно
  быть проведено в этой области, чтобы получить решения калибровочной теории для конкретных
  дефектов>>~\cite{sahoo-gauge-I}.

  \subsection{Формулировка и обоснование проблемы}

  Из сказанного выше очевидным образом вытекает основная проблематика калибровочных моделей дефектов
  в кристаллических телах и, в частности, в полупроводниках. Главное противоречие заключается в
  следующем: структурные дефекты (особенно дислокации) являются давно открытыми, легко наблюдаемыми
  и чрезвычайно подробно эмпирически изученными явлениями, но при этом до сих пор нет единой
  удовлетворительной теории, способной описать их все. Существующие теории либо могут описать только
  один вид дефектов (и то не полностью), либо их описание неточно и плохо согласуется с
  экспериментом, либо они описывают лишь статистически средние величины и не могут описать движение
  отдельных дефектов. К тому же большинство из них весьма сложны математически, из-за чего их
  результаты сложно интерпретировать, даже если они верны~--- другими словами, ощущается серьёзная
  нехватка средств графической визуализации для таких теорий.

  Актуальность проблемы связана с обоснованным выше критическим значением кинематики и динамики
  структурных дефектов как для предсказания электронных свойств полупроводниковых приборов, а также
  для более общей задачи~--- описания процессов пластичности в металлах и других кристаллических
  веществах. Доработка существующих теорий и разработка средств визуализации для них может иметь как
  фундаментальное, так и прикладное значение, в частности, при проектировании полупроводниковых
  приборов.

  \section{Теоретико-методологические основания исследования}

  Выше была рассмотрена история и проблематика моделей дефектов в кристаллических телах. Рассмотрим
  теперь теоретико-методологические основания планируемого диссертационного исследования по этой теме.

  С точки зрения методологии существует теоретический и эмпирический уровни научного познания.
  Основой эмпирических методов являются чувственное познание (ощущение, восприятие, представление) и
  данные приборов. К числу этих методов относятся: наблюдение, эксперимент, измерение и сравнение.
  Собственно теоретические методы опираются на рациональное познание (понятие, суждение,
  умозаключение) и логические процедуры вывода; сюда относится анализ, синтез, идеализация,
  классификация и др.

  Дополняет эти <<классические>> методологии такой подход, как моделирование, особенно всё шире
  применяемое в настоящее время компьютерное моделирование.
  %
  \term{Моделирование}~--- это изучение объекта (оригинала) путём создания и исследования его копии
  (модели), замещающей оригинал с определённых сторон, интересующих познание. Модель всегда
  соответствует объекту-оригиналу в тех свойствах, которые подлежат изучению, но в то же время
  отличается от него по другим, не интересующим исследователя признакам, что делает модель удобной
  для исследования объекта~\cite{frolov-phylos}.

  Можно сказать, что компьютерное моделирование является объединением теоретических и эмпирических
  методов: для построения математической модели необходимы теоретические основания, но сам процесс
  обработки результатов моделирования и сравнения их с экспериментом является эмпирическим. Это и
  делает моделирование особенно ценным. Причём в случае компьютерного моделирования

  В качестве основной парадигмы в планируемом диссертационном исследовании используется описанный
  выше калибровочный подход к модели дефектов в полупроводниках. В частности, описание искривления
  кристаллической решётки из-за дефектов будет производиться с помощью математического аппарата
  неримановой геометрии пространств афинной связности. Планируется уделить в работе большое внимание
  разработке средств визуализации такой геометрии с помощью трёхмерной компьютерной графики,
  использовав существующие наработки по 3D-визуализации и виртуальной реальности~\cite{nia-mnsk-11}.

  Объектом планируемой диссертации являются структурные дефекты в кристаллических телах, а
  предметом~--- способы моделирования и визуализации таких дефектов на основе калибровочных теорий.
  Целью этой работы является разработка на основе имеющихся теорий такой модели структурных дефектов
  в полупроводниках, которая позволяет рассчитывать и визуализировать дислокации и их свойства, а
  её результаты согласовываются с экспериментом. Для достижения этой цели поставлены следующие задачи.
  \begin{enumerate}
    \item Анализ существующих калибровочных теорий структурных дефектов в кристаллах.
    \item Создание компьютерной модели дефектов в полупроводниках.
    \item Разработка средств визуализации результатов расчётов этой модели.
    \item Верификация результатов моделирования.
  \end{enumerate}

  \sectiontoc{Заключение}

  Был приведён обзор такой области науки, как физика структурных дефектов в полупроводниках.
  Показано, что эта область имеет глубокий философский, историко"=научный и теоретико"=методологический
  потенциал, а также практическую актуальность. Существует множество подходов к теоретическому
  осмыслению явления дислокаций и других структурных дефектов, при этом создание единой и точной
  теории пока остаётся открытой проблемой, так же как и визуализация результатов такой теории.

  В планируемом диссертационном исследовании после анализа существующих наработок в этой области
  должна быть разработана компьютерная модель структурных дефектов и средства визуализации её
  результатов на основе подхода, известного как калибровочные теории поля, и математического
  аппарата дифференциальной геометрии. Этот мощный аппарат является многообещающим для решения
  указанной научной проблемы.

  \PrintBibliography

\end{document}
