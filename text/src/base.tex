\documentclass[a4paper, 14pt, titlepage]{extarticle}
  \usepackage[bib,figtable,newpage,titlegost]{mypreamble} % моя преамбула
  \usepackage[babel,protrusion=true,expansion]{microtype}

  \newcommand{\thetitle}{Модель структурных дефектов в полупроводниках}
  \newcommand{\theauthor}{Иван Новиков}

  \author{\theauthor,\\ кафедра физики и информационных систем КубГУ}
  \title{\thetitle}
  \date{\today{} \currenttime}

  \SetPdfTitleAndAuthor{\thetitle}{\theauthor}

\begin{document}

%----------------------- титульный лист ------------------------

  \maketitle
  \newpage

  \thispagestyle{empty}
  \begin {center}
  \small{МИНИСТЕРСТВО ОБРАЗОВАНИЯ И НАУКИ РОССИЙСКОЙ ФЕДЕРАЦИИ}\\
  Федеральное государственное бюджетное образовательное учреждение\\
  высшего профессионального образования\\
  «КУБАНСКИЙ ГОСУДАРСТВЕННЫЙ УНИВЕРСИТЕТ»\\
  (ФГБОУ ВПО «КубГУ»)

  Физико-технический факультет

  \vspace {1cm}

  Кафедра физики и информационных систем

  \vspace {5cm}

  \textbf{РЕФЕРАТ \\ по предмету <<Логика и методология научного познания>>}

  \vspace {0.5cm}

  \textbf{ \large \scshape \thetitle } \todo{тема реферата?}

  \vspace {1.5cm}

  \begin{flushleft}
    Работу выполнил \hrulefill{}  (Новиков Иван Александрович)\\
    Направление аспирантуры 01.04.10 Физика полупроводников

    Проверил\\
    профессор, д-р филос. наук \hrulefill{} (П.\,Е.\,Бойко)
    \\[-3mm]{\footnotesize\centering (подпись, дата)\par}

    Научный руководитель\\
    профессор, д-р физ.-мат. наук \hrulefill{} (Н.\,М.\,Богатов)
    \\[-3mm]{\footnotesize\centering (подпись, дата)\par}

  \end{flushleft}

  \vfill

  Краснодар \the\year
  \end {center}

%------------------------- содержание -------------------------

  \microtypesetup{protrusion=false} % отключить protrusion для содержания
    \clearpage
    \tableofcontents
  \microtypesetup{protrusion=true}

%-------------------------- введение --------------------------
  \sectiontoc{TEMP: Доклад}

  \subsection{Предполагаемый план}

  \begin{enumerate}
    \item Введение: Основы
      \begin{enumerate}
        \item Терминология \label{enu:terms}
        \item Что такое полупроводник, какие бывают дефекты (из Матаре) \label{enu:semicond}
        \item Особенно остановиться на дислокациях.
      \end{enumerate}
    \item Разные подходы к численному моделированию \emph{(Очень базово)}
      \begin{enumerate}
        \item можно решать дифуры точно
        \item можно решать дифуры численно. \label{enu:numeric}
        \item упрощённые алгоритмы          \label{enu:simple}
      \end{enumerate}
    \item Геометрический подход к моделированию.
      \begin{enumerate}
        \item Краткий экскурс, как я к этому пришёл,\dots
        \item \dots Линейные преобраз-я: про мой бакалаврский диплом/умник\dots
        \item \dots Искривление пространства: gauge theory
      \end{enumerate}
    \item Gauge theories~--- калибровочные теории \label{enu:gauge}
      \begin{enumerate}
        \item \todo{to be continued\dots}
      \end{enumerate}
    \item Стандартные пункты введения в диссер
      \begin{enumerate}
        \item Актуальность темы\\ (напр, незаконченность теорий + нехватка визуализации)
        \item Формулировка и степень разработанности проблемы
        \item Объект, предмет, цель, задачи
        \item Примерная концепция, новизна
      \end{enumerate}
  \end{enumerate}

  %\subsection{Где философия?}

  %Какие из вышеперечисленных пунктов относятся к \emph{логическому, философскому, историческому
  %введению в тему диссертации}?

  %\begin{itemize}
  %  \item \ref{enu:terms}, \ref{enu:semicond}~--- раскрытие понятий
  %  \item \ref{enu:numeric}, \ref{enu:simple}~--- противоречие точно$\rightleftharpoons$быстро.
  %  \item \ref{enu:gauge}~--- история вопроса
  %\end{itemize}
  %
  %\todo{а ещё?}

  \sectiontoc{Введение}\label{sec:intro}

  Полупроводники в настоящее время имеют огромное значение, поскольку лежат в основе всей
  современной электроники и компьютерной техники, а также применяются в солнечной энергетике и
  оптоэлектронике. При этом изучение структурных дефектов в полупроводниках представляет особый
  интерес, с~одной стороны, поскольку они могут существенно влиять на электрические свойства
  полупроводниковых приборов (как отрицательно, так и положительно), а с другой~--- по причине их
  участия в механических процессах, таких как пластичность, причём последнее имеет место в широком
  смысле: не только для полупроводников, но и для других кристаллических веществ.

  Раскроем основные понятия, возникающие при рассмотрении темы структурных дефектов в полупроводниках:
  полупроводник, его структура, дефекты в этой структуре и, наконец, дислокации как наиболее важный
  вид дефектов.

  \paragraph{Полупроводник.} Термин \term{полупроводник} возникает при рассмотрении такого важного
  свойства материи, как способность проводить электрический ток. По отношению к этому свойству все
  вещества можно разделить на три класса: металлы, полупроводники и диэлектрики. Выбор критерия
  такой классификации не является очевидным. Казалось~бы, проще всего использовать в качестве такого
  критерия удельное сопротивление вещества.

  \begin{mytable}{Диапазоны удельного сопротивления $\rho$ классов веществ}{tab:rho}
    \begin{tabular}{r|l}
      Класс вещества & $\rho$, Ом$\cdot$см \\
      \hline
      Проводники     & $10^{-6}$ .. $10^{-4}$\\
      Полупроводники & $10^{-4}$ .. $10^{10}$\\
      Диэлектрики    & $> 10^{10}$\\
    \end{tabular}
  \end{mytable}

  Однако, как следует из таблицы \ref{tab:rho}, диапазоны удельного сопротивления различных классов
  перекрываются. Например, сернистый кадмий (полупроводник) в~зависимости от технологии изготовление
  может иметь удельное сопротивление как ниже, чем нихром (металлический сплав), так и выше, чем
  стекло (диэлектрик). Следовательно, одно только значение удельного сопротивления не может служить
  однозначным критерием.

  Решение этого вопроса можно найти, если проследить, как изменяется удельное сопротивление при
  понижении температуры. Сопротивление металлов при этом уменьшается: в~них свободные электроны
  присутствуют независимо от температуры, а подвижность увеличивается за счёт ослабления тепловых
  колебаний решётки. У полупроводников же, напротив, удельное сопротивление повышается при снижении
  температуры: свободные носители заряда возникают в нём за счёт теплового возбуждения
  (\term{равновесные} носители), а~также при освещении и ядерном облучении (\term{неравновесные}).
  Этот процесс образования носителей (как равновесных, так и неравновесных) сильно зависит от
  структуры полупроводникового материала и от наличия примесей.

  С~учётом вышесказанного, можно дать следующее определение. \term{Полупроводник}~--- это такой
  материал, который при комнатной температуре имеет удельное сопротивление от $10^{-6}$ до $10^{-4}$
  Ом$\cdot$см, которое сильно зависит от
  \begin{enumerate*}[label=\asbuk*)]
    \item структуры вещества
    \item вида и количества примесей,
    \item внешних условий: температуры, освещения, облучения ядерными частицами, электромагнитного
      поля~\cite{shalimova-semiconductors}.
  \end{enumerate*}

  \paragraph{Структура полупроводника.} Физические свойства полупроводников определяются, главным
  образом, тем, что они являются кристаллами. С этим связаны два основных понятия.

  \term{Кристаллическая решётка}~--- присущее кристаллам регулярное расположение частиц (атомов, их
  ядер, ионов, молекул, электронов), характеризующееся периодичной повторяемостью в трёх
  измерениях~\cite{physenc-crystall}; вспомогательный геометрический образ, вводимый для анализа
  строения кристалла.

  \term{Кристаллическая структура}~--- такая совокупность атомов, в которой с каждой точкой
  кристаллической решётки связана определённая группа атомов, называемая \term{мотивной единицей}. Можно считать,
  что структура возникает в результате синтеза решётки и мотивной единицы, в результате размножения
  мотивной единицы группой трансляции \cite{wiki-cryst-struct}.
  Обычно, говоря о кристаллической структуре, подразумевают среднее во времени расположение атомных
  ядер (так называемая статическая модель); более полная информация включает сведения об амплитудах и частотах
  колебаний атомов (динамическая модель), а также о распределении электронной плотности в межъядерном
  пространстве \cite{chemenc-crystall}.

  \paragraph{Структурные дефекты.} \term{Дефекты} в кристаллах~--- устойчивые нарушения правильного
  расположения атомов или ионов в узлах кристаллической решётки, соответствующего минимуму потенциальной
  энергии кристалла. Иначе говоря, структурные дефекты, или несовершенства~--- это отклонения
  расположения атомов от идеальной кристаллической структуры. Существуют большое количество
  различных видов дефектов, их можно классифицировать по размерности.
  \begin{enumerate}
    \item Нулевой размерности~--- точечные дефекты, например, вакансии и междоузельные атомы.
    \item Одномерные~--- линейные дефекты~--- дислокации.
    \item Двумерные~--- поверхностные дефекты, например, границы зёрен.
    \item Трёхмерные~--- объёмные дефекты, например, поры \cite{matare-defects}.
  \end{enumerate}

  \paragraph{Дислокации}--- дефекты кристаллической решётки, искажающие правильное расположение
  атомных (кристаллографических) плоскостей. Два базовых типа: краевая и винтовая дислокации.
  \term{Краевая дислокация} соответствует атомной плоскости, обрывающейся внутри кристалла (рис.~\ref{fig:dislocation}).
  \term{Винтовую дислокацию} можно представить себе как результат сдвига на период решётки одной
  части кристалла относительно другой вдоль некоторой полуплоскости параллельно её краю, играющему
  роль оси дислокации.

  % TODO вернуть нормальный масштаб
  \includefigure[width=0.2\textwidth]{Краевая дислокация в кубической решётке}{dislocation}
  % TODO рисунок с винтовой дислокацией

  % Рассмотрев основные понятия, вкратце отметим историю изучения дефектов в полупроводниках.
  В начала ХХ~в. теория дислокации появилась вне физики полупроводников как свойства произвольных
  кристаллов. Механическая теория дислокаций развивалась как раздел металловедения, благодаря ей
  удалось объяснить механизм пластических деформаций. После того, как началось изучение
  полупроводников для использования их в электронных приборах, были также изучены электрические
  свойства дислокаций~\cite{matare-defects}. Изначально теории дислокаций основывались на
  классической теории упругости. Однако, в связи с успехом калибровочных теорий в таких областях,
  как квантовая электродинамика и общая теория относительности были предприняты попытки создания
  калибровочной теории дефектов в твёрдых телах.

  \section{Различные подходы к численному моделированию}

  \section{Геометрические модели физических процессов}

  \section{Калибровочные теории дефектов}

  \textbf{\textcolor{red}{\large NB: здесь и далее лишь наброски для доклада}}

  Сначала - откуда всё пришло.

  \sectiontoc{Заключение}

  Таким образом, \dots

  \PrintBibliography

\end{document}
