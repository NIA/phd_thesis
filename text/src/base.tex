\documentclass[a4paper, 14pt, titlepage]{extarticle}
  % TODO почистить преамбулу от неиспользуемых вещей
  \usepackage{cmap}
  \usepackage[hidelinks,pdftex,unicode]{hyperref}
  \usepackage[T2A]{fontenc}
  \usepackage[utf8]{inputenc}
  \usepackage[english,russian]{babel}
  \usepackage{indentfirst}
  \usepackage{cite}
  \usepackage[usenames,dvipsnames]{xcolor}
  \usepackage[pdftex]{graphicx}
    \graphicspath{{../img/}{../../img/}}
  \usepackage{subfig}
  \usepackage{numprint}
  \usepackage[left=30mm,right=15mm,top=20mm,bottom=20mm,bindingoffset=0cm]{geometry}
  \usepackage{datetime}
  \usepackage{setspace} % для иных отступов в tikz-картинках
  \usepackage{tikz}
    \usetikzlibrary{positioning,fit,shapes,calc,backgrounds}
    % Цветовая тема:
    \colorlet{color1}{violet}         \colorlet{color1bg}{violet!15}
      \colorlet{color1a}{violet!70!Blue}\colorlet{color1abg}{violet!20}
    \colorlet{color2}{cyan!80!Blue}   \colorlet{color2bg}{cyan!15}
    \colorlet{color3}{Emerald}        \colorlet{color3bg}{green!15}
    \colorlet{color4}{Orange}         \colorlet{color4bg}{orange!20}
    \colorlet{color5}{OliveGreen}     \colorlet{color5bg}{green!15!yellow!15}
    \colorlet{color0}{Gray}           \colorlet{color0bg}{Gray!10}
    % Определяет стили theme1, theme1a, theme2...
    \newcommand{\deftheme}[1]{\tikzstyle{theme#1}  = [draw=color#1,  fill=color#1bg]}
    \deftheme{0}\deftheme{1}\deftheme{1a}\deftheme{2}\deftheme{3}\deftheme{4}\deftheme{5}
  \usepackage[babel,protrusion=true,expansion]{microtype}
  \usepackage[labelsep=endash]{caption} % тире в подписях рисунков, таблиц
  \usepackage{listings}
    \lstloadlanguages{C++}
    \lstset{
      language=C++,          % везде язык C++
      extendedchars=true,    % включаем не латиницу
      frame=tb,              % рамка сверху и снизу
      basicstyle=\small, % основной шрифт
      commentstyle=\itshape\color{black!50}, % шрифт комментариев
      gobble=6,              % по умолчанию удалять 6 пробелов в начале строки (отступы от \begin)
      texcl=true,            % комментарии - TeX, позволяет использовать русские буквы
      xleftmargin=1.5cm,       % дефолтные поля по 2см,
      xrightmargin=1.5cm,      % чтобы смотрелось как рисунок
      morekeywords={nullptr, override,
        signals, slots, emit}, % ключевые слова Qt MOC и C++11
      breaklines=true,       % включить авто переносы
      breakatwhitespace=false, % разрешить переносить не только по пробелу
      showstringspaces=false % не подчёркивать пробелы в строках
    }
    % для кода в строчку
    \newcommand{\inlinecode}[1]{\lstinline[basicstyle=\ttfamily]{#1}}
  \usepackage{totcount} % для посчёта полного числа рисунков итп
    % подсчёт рисунков:
    \regtotcounter{figure}
    % полное число страниц
    \regtotcounter{page}
    % подсчёт источников:
    \newtotcounter{citnum}
    \let\oldbibitem\bibitem
    \renewcommand\bibitem{\stepcounter{citnum}\oldbibitem}
    % правильный подсчёт страниц (от введения до заключения, не считая содержание и список источников
    % USAGE: Поместить \firstPageHere перед первой страницей, а \lastPageHere на последней странице.
    %        После этого можно использовать \total{pagenum} в любом месте документа
    \newtotcounter{pagenum}
    \newcounter{firstpage}
    \newcommand\firstPageHere{\setcounter{firstpage}{\value{page}}}
    \newcommand\lastPageHere{\setcounter{pagenum}{\value{page} - \value{firstpage}}}
  \usepackage{titlesec}
    \titlespacing{\paragraph}{\parindent}{0pt}{1em} % \paragraph без вертикального пропуска, но с абзацным отступом
    % Все заголовки: обычный размер шрифта, абзацный отступ, пропуск одной строки до и после заголовка
    \titleformat*{\section}{\normalfont\normalsize\bfseries}
    \titlespacing{\section}{\parindent}{0pt}{\baselineskip}
    \titleformat*{\subsection}{\normalfont\normalsize\bfseries}
    \titlespacing{\subsection}{\parindent}{\baselineskip}{\baselineskip}
    \titleformat*{\subsubsection}{\normalfont\normalsize\bfseries}
    \titlespacing{\subsubsection}{\parindent}{\baselineskip}{\baselineskip}
  \usepackage{enumitem}
    % Убираем пропуски вокруг списков и между их элементами
    \setlist{nosep}
    % Фикс для подпунктов а), б) ...
    \AddEnumerateCounter{\Asbuk}{\@Asbuk}{Ы}
    \AddEnumerateCounter{\asbuk}{\@asbuk}{ы}
    % Отдельно фиксим пропуски в библиографии
    \let\oldthebibliography\thebibliography
    \renewcommand\thebibliography[1]{
      \oldthebibliography{#1}
      \setlength{\parskip}{0pt}
      \setlength{\itemsep}{0pt plus 0.3ex}
    }
  \frenchspacing

  \DeclareSymbolFont{T2Aletters}{T2A}{cmr}{m}{it} % кириллица в формулах курсивом

  \addto\captionsrussian{
    \renewcommand\contentsname{\centering Содержание}
    \renewcommand\figurename{Рисунок}
    \renewcommand\refname{Список использованных источников}
    % перекрываю \refname, чтобы список литературы сам добавлял себя в оглавление
    \let\oldrefname\refname
    \renewcommand\refname{\addcontentsline{toc}{section}{\oldrefname}\centering \oldrefname}
  }

  % Ненумерованный section, но добавленный в оглавление
  \newcommand\sectiontoc[1]{\section*{\centering #1}\addcontentsline{toc}{section}{#1}}

  \newcommand{\underscore}[1]{\hbox to#1{\hrulefill}}
  \newcommand{\todo}[1]{\textbf{\textcolor{red}{TODO: #1}}}
  \newcommand{\note}[1]{\textit{Примечание: #1}}
  \newcommand{\eng}[1]{\foreignlanguage{english}{#1}}

  % выделение термина в определении (и не только)
  \newcommand{\term}[1]{\emph{#1}}

  % обёртка с моими настройками поверх figure:
  % \begin{myfigure}{подпись}{fig:label} ... \end{myfigure}
  \newenvironment{myfigure}[2]%
    {\pushQED{\caption{#1} \label{#2}} % push caption & label
     \begin{figure}[!htb]\centering } %
    {  \popQED % pop caption & label
     \end{figure}}

  % вставка картинки: \figure[params]{подпись}{file}
  % создаёт label вида fig:file
  \newcommand{\includefigure}[3][]{
    \begin{myfigure}{#2}{fig:#3}
      \includegraphics[#1]{#3}
    \end{myfigure}
  }

  % вставка subfigure внутри myfigure:
  % \subfigure[params]{подпись}{file}
  \newcommand{\subfigure}[3][]{
    \subfloat[#2]{\label{fig:#3}\includegraphics[#1]{#3}}
  }

  \renewcommand{\le}{\leqslant} % <= с наклонной нижней перекладиной
  \renewcommand{\ge}{\geqslant} % >= с наклонной нижней перекладиной

  \linespread{1.3}

  % русские буквы для списков и частей рисунка
  \renewcommand{\theenumii}{(\asbuk{enumii})}
  \renewcommand{\labelenumii}{\asbuk{enumii})}
  \renewcommand{\thesubfigure}{\asbuk{subfigure}}

  % разделы с новой страницы
  \let\oldsection\section
  \renewcommand{\section}{\newpage\oldsection}

  \setcounter{tocdepth}{3} % глубина оглавления

  \hyphenation{англ} % убрать перенос в этом сокращении

  % алиас и настройки для numprint
  \newcommand{\num}[1]{\numprint{#1}}
  \npthousandsep{\,}
  \npthousandthpartsep{}
  \npdecimalsign{,}

  % нумерация в списке использованных источников в виде 1. 2. 3., а не [1] [2] [3]
  \makeatletter
  \renewcommand\@biblabel[1]{#1.}
  \makeatother

  \newcommand{\checkdate}[3]{(дата обращения: \formatdate{#1}{#2}{#3})}

  \newcommand{\thetitle}{Модель структурных дефектов в полупроводниках}
  \newcommand{\theauthor}{Иван Новиков}

  \author{\theauthor,\\ кафедра физики и информационных систем КубГУ}
  \title{\thetitle}
  \date{\today{} \currenttime}

  \hypersetup{
    pdfinfo={
      Title = {\thetitle},
      Author = {\theauthor},
      Subject = {}
    }
  }

\begin{document}

%----------------------- титульный лист ------------------------

  \maketitle
  \newpage

  \thispagestyle{empty}
  \begin {center}
  \small{МИНИСТЕРСТВО ОБРАЗОВАНИЯ И НАУКИ РОССИЙСКОЙ ФЕДЕРАЦИИ}\\
  Федеральное государственное бюджетное образовательное учреждение\\
  высшего профессионального образования\\
  «КУБАНСКИЙ ГОСУДАРСТВЕННЫЙ УНИВЕРСИТЕТ»\\
  (ФГБОУ ВПО «КубГУ»)

  Физико-технический факультет

  \vspace {1cm}

  Кафедра физики и информационных систем

  \vspace {5cm}

  \textbf{РЕФЕРАТ \\ по предмету <<Логика и методология научного познания>>}

  \vspace {0.5cm}

  \textbf{ \large \scshape \thetitle }

  \vspace {1.5cm}

  \begin{flushleft}
    Работу выполнил \hrulefill{}  (Новиков Иван Александрович)\\
    Направление аспирантуры 01.04.10 Физика полупроводников

    Проверил\\
    профессор, д-р филос. наук \hrulefill{} (П.\,Е.\,Бойко)
    \\[-3mm]{\footnotesize\centering (подпись, дата)\par}

    Научный руководитель\\
    профессор, д-р физ.-мат. наук \hrulefill{} (Н.\,М.\,Богатов)
    \\[-3mm]{\footnotesize\centering (подпись, дата)\par}

  \end{flushleft}

  \vfill

  Краснодар \the\year
  \end {center}

%------------------------- содержание -------------------------

  \microtypesetup{protrusion=false} % отключить protrusion для содержания
    \clearpage
    \tableofcontents
  \microtypesetup{protrusion=true}

%-------------------------- введение --------------------------
  \firstPageHere

  \sectiontoc{TEMP: Доклад}

  \subsection{Предполагаемый план}
  \begin{enumerate}
    \item Введение: Основы
      \begin{enumerate}
        \item Терминология \label{enu:terms}
        \item Что такое полупроводник, какие бывают дефекты (из Матаре) \label{enu:semicond}
        \item \todo{Уже здесь актуальность?}
      \end{enumerate}
    \item Разные подходы к численному моделированию \emph{(Очень базово)}
      \begin{enumerate}
        \item можно решать дифуры точно
        \item можно решать дифуры численно. \label{enu:numeric}
        \item упрощённые алгоритмы          \label{enu:simple}
      \end{enumerate}
    \item Геометрический подход к моделированию.
      \begin{enumerate}
        \item Краткий экскурс, как я к этому пришёл,\dots
        \item \dots Линейные преобраз-я: про мой бакалаврский диплом/умник\dots
        \item \dots Искривление пространства: gauge theory
      \end{enumerate}
    \item Gauge theories~--- калибровочные теории \label{enu:gauge}
      \begin{enumerate}
        \item \todo{to be continued\dots}
      \end{enumerate}
    \item Стандартные пункты введения в диссер
      \begin{enumerate}
        \item Актуальность темы\\ (напр, незаконченность теорий + нехватка визуализации)
        \item Формулировка и степень разработанности проблемы
        \item Объект, предмет, цель, задачи
        \item Примерная концепция, новизна
      \end{enumerate}
  \end{enumerate}

  \subsection{Где философия?}

  Какие из вышеперечисленных пунктов относятся к \emph{логическому, философскому, историческому
  введению в тему диссертации}?

  \begin{itemize}
    \item \ref{enu:terms}, \ref{enu:semicond}~--- раскрытие понятий
    \item \ref{enu:numeric}, \ref{enu:simple}~--- противоречие точно$\rightleftharpoons$быстро.
    \item \ref{enu:gauge}~--- история вопроса
  \end{itemize}
  
  \todo{а ещё?}

  \sectiontoc{Введение}\label{sec:intro}

  \section{Измерительные информационные системы}\label{sec:iis}

  Тут вступление к разделу.

  % ------ begin section ------------

  Тут текст раздела. Можно сослаться на старый диплом \cite{nia-mnsk-11}.


  % ------ end section ------------

  \paragraph{Выводы к разделу.} Тут выводы

  \section{Ещё один клёвый раздел}\label{sec:hardware}

  Тут вступление к разделу.

  % ------ begin section ------------

  Тут текст раздела.

  % ------ end section ------------

  \paragraph{Выводы к разделу.} Тут выводы

  \sectiontoc{Заключение}

  Результаты, бла, бла, бла.

  %\lastPageHere

  \begin{flushleft}
    \bibliographystyle{../../biblio/ugost2003} % Или использовать ugost2008 для нового ГОСТа
    \bibliography{../../biblio/my,../../biblio/own}
  \end{flushleft}
\end{document}
