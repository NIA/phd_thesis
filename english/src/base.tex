\documentclass[a4paper, 14pt, titlepage]{extarticle}
  \usepackage[bib,math,newpage,titlegost]{mypreamble} % моя преамбула
  \usepackage[babel,protrusion=true,expansion]{microtype}

  \newcommand{\thetitle}{Современные геометрические модели дислокаций и пластичности кристаллов (реферат по иностранному языку)}
  \newcommand{\theauthor}{Иван Новиков}

  \author{\theauthor,\\ кафедра физики и информационных систем КубГУ,\\ направление аспирантуры 01.04.10}
  \title{\thetitle}
  \date{\today{} \currenttime}

  \SetPdfTitleAndAuthor{\thetitle}{\theauthor}

  % некоторые распространённые индексы в формулах:
  \newcommand{\tot}{\textup{t}}    % - полная (плотность)
  \newcommand{\plast}{\textup{pl}} % - пластическая (деформация)
  \newcommand{\II}{\textup{II}}    % - второго порядка (тензор \alpha)

\begin{document}

%----------------------- титульный лист ------------------------

  \maketitle

%------------------------- содержание -------------------------

  \microtypesetup{protrusion=false} % отключить protrusion для содержания
    \clearpage
    \tableofcontents
  \microtypesetup{protrusion=true}

%-------------------------- введение --------------------------

  \sectiontoc{Введение}\label{sec:intro}

  Полупроводники в настоящее время имеют огромное значение, поскольку лежат в основе всей
  современной электроники и компьютерной техники, а также применяются в солнечной энергетике и
  оптоэлектронике. При этом изучение структурных дефектов в полупроводниках представляет особый
  интерес, с~одной стороны, поскольку они могут существенно влиять на электрические свойства
  полупроводниковых приборов (как отрицательно, так и положительно), а с другой~--- по причине их
  участия в механических процессах, таких как пластичность, причём последнее имеет место в широком
  смысле: не только для полупроводников, но и для других кристаллических веществ.

  Среди различных типов дефектов кристаллической решётки особенный интерес представляют
  \term{дислокации}~--- линейные (одномерные) дефекты, оказывающие наибольшее влияние на электрические и
  механические свойства кристалла. Это такие дефекты, которые искажают правильное расположение
  атомных (кристаллографических) плоскостей. Два базовых типа: краевая и винтовая дислокации.
  \term{Краевая дислокация} соответствует атомной плоскости, обрывающейся внутри кристалла.
  \term{Винтовую дислокацию} можно представить себе как результат сдвига одной
  части кристалла относительно другой вдоль некоторой полуплоскости параллельно её краю,
  в~результате чего атомные плоскости винтообразно изгибаются.

  Помимо существенного влияния на свойства кристалла, есть и другая причина, по которой
  дислокации представляют интерес для учёных. Несмотря на то, что данное явление широко изучено
  экспериментально, до сих пор не существует единой общепринятой теории, количественно описывающей
  возникновение и динамику дислокаций и хорошо согласующейся с экспериментом.

  В реферате рассмотрены несколько последних разработок в области теории дислокаций,
  опирающихся на разные подходы, но имеющих общую цель: создание полной и пригодной для
  компьютерного моделирования теории, согласующейся с экспериментом и позволяющей описывать
  связанные с дислокациями явления, такие как пластичность.

  \section{Теория пластичности и динамики дислокаций на основе механики сплошных сред}

  В работе~\cite{hochrainer-cdd} предлагается строить непрерывную теорию динамики деформаций и
  пластичности на основе классических уравнений динамики деформируемых сплошных сред.

  Многие явления, связанные с пластичностью металлов (такие как наклёп), могут быть напрямую
  объяснены за счёт взаимодействия и размножения дислокаций. Тем не менее, как отмечают авторы
  работы~\cite{hochrainer-cdd}, большинство существующих теорий не охватывают перемещения дислокаций в полной мере.
  Дело в том, что для традиционных инженерных приложений эта проблема не является существенной.
  Более того, обычным подходом для таких приложений вообще является отказ от рассмотрения дислокаций
  в пользу простой параметризации материальных уравнений, подгоняемой под результаты механических
  испытаний. Однако, при переходе к структурам малого масштаба локальные материальные уравнения
  перестают работать, так как не могут объяснить наблюдаемую зависимость от размеров.

  Вышесказанное объясняет потребность в непрерывной статистической теории дислокаций и пластичности. При
  формулировании же такой теории возникает фундаментальная сложность, связанная с выражением
  расположения и движения дислокаций в виде непрерывных величин. Авторы утверждают, что в своих
  прежних работах им удалось преодолеть эту трудность путём рассмотрения абстрактного
  конфигурационного пространства более высокого измерения, включающего дополнительно координаты,
  характеризующие направление дислокационной линии. Но получаемая таким образом теория является либо
  неточной (при малой размерности пространства), либо непригодной для эффективного компьютерного
  моделирования (при высокой размерности). Поэтому в работе~\cite{hochrainer-cdd} предпринимается попытка
  преодолеть эти трудности систематически путём мультипольного разложения функции плотности.
  Получаемые при этом уравнения ложатся в основу теории поля, которой авторы дают название
  \term{непрерывная динамика дислокаций (НДД)}.

  \subsection{Непрерывная динамика дислокаций} % TODO можно ли лучше перевести Continuum Dislocation Dynamics (CDD)?

  Необратимая (пластическая) деформация возникает в результате \term{скольжения дислокаций}. Для описания
  этого процесса традиционно используются следующие термины и характеризующие их численные величины. Скольжение
  происходит вдоль определённых плоскостей, называемых \term{плоскостями скольжения} и
  характеризуемых нормалью $\vect{n}$, и в определённых направлениях, называемых \term{направлениями
  скольжения}, задаваемых вектором $\vect{m}$. Дислокации характеризуются \term{вектором Бюргерса}
  $\vect{b} = b \vect{m}$, выражающим сдвиг в плоскости скольжения, и направлением дислокационной
  линии $\vect{l}$.  Плотность дислокаций задаётся двумя классическими величинами: длиной
  дислокационной линии в единице объёма $\rho_\tot$~--- \term{полной плотностью дислокаций} и
  \term{тензором плотности дислокаций} $\matx{\alpha}$. % определения через $\matx{\beta}^\plast$ и $\vect{b}$ опущены
  Авторы работы~\cite{hochrainer-cdd} предлагают строить свою теорию на основе аналога этого тензора
  в вышеупомянутом пространстве большей размерности~--- \term{тензора плотности дислокаций второго
  порядка} $\matx{\alpha}^\II$, получаемого за счёт добавления угла $\phi$ между векторами $\vect{l}$ и
  $\vect{b}$ в качестве независимого параметра.

  Для вывода новых уравнений и построения тензора $\matx{\alpha}^\II$ авторы выбирают определённую
  систему координат (в которой $\vect{b} = (b,0,0)$) и рассматривают две функции плотности.
  Первая, $\rho(p,\phi)$, выражает количество дислокаций, проходящих через единичный элемент площади в
  точке $p$ в направлении $\vect{l}(\phi)$. Вторая, \term{плотность кривизны} $q(p,\phi)$,
  характеризует изменение направления вдоль дислокационной линии.
  Чтобы преодолеть главную проблему такого подхода~--- повышенные вычислительные затраты из-за
  увеличенной размерности~--- авторы применяют разложение этих функций в ряд Фурье и используют
  коэффициенты этого разложения в качестве переменных упрощённой теории НДД.
  Первые из этих коэффициентов представляют собой хорошо известные и упомянутые ранее величины: это
  полные плотности дислокаций ($\rho_\tot$) и кривизны ($q_\tot$), вектор плотности дислокаций
  $\vect{\kappa}(p) = (\kappa^1,\kappa^2,0)$ и классический тензор плотности дислокаций $\matx{\alpha}$.

  При построении эволюционных уравнений для усреднённых величин возникает бесконечная цепочка
  зависимостей: коэффициенты Фурье более низких порядков выражаются через коэффициенты более высоких
  порядков. Чтобы получить конечный набор уравнений, требуется аппроксимация путём замыкания на
  определённом порядке. В своей работе авторы стремятся замкнуть систему на наименьшем возможном
  порядке, то есть получить описание динамики дислокаций с наименьшим количеством свободных
  переменных. По их мнению, для материалов с изотропной подвижностью дислокаций такой минимальный
  набор~--- это величины $\rho_\tot$, $q_\tot$ и $\vect{\kappa}$. В результате выкладок они приходят
  к уравнению
  \[
    \partial_t \rho_\tot = - \nabla \cdot \left( \nu \vec{\kappa}^\bot \right) + \nu q_\tot ,
  \]
  где $\vect{\kappa}^\bot = (\kappa^2, -\kappa^1, 0)$, то есть $\vect{\kappa}^\bot \bot \vect{\kappa}$,
  а $\nu$~--- скорость дислокации.

  % The evolution equation (Eq. 2.5) constitutes the first major result of this paper and can be directly interpreted as follows:
  \todo{Здесь будет остальная часть описания теории.}

  \subsection{Результаты моделирования}

  \todo{Здесь будут кратко перечислены результаты и выводы из них.}

  \section{Полевая теория многомасштабной пластичности}

  В работе~\cite{hasebe-ftmp} предпринимается попытка объединить наработки калибровочной,
  дифференциально-геометрической и квантовой теорий поля в обобщённую теорию~--- так называемую
  полевую теорию многомасштабной пластичности.

  \section{Калибровочная теория дислокаций в двумерных нематиках}

  В работе~\cite{liu-nematic} \dots

  \sectiontoc{Заключение}

  Таким образом, \dots

  \PrintBibliography

\end{document}
