\documentclass[a4paper, 14pt, titlepage]{extarticle}
  \usepackage[bib,newpage,titlegost]{mypreamble} % моя преамбула
  \usepackage[babel,protrusion=true,expansion]{microtype}

  \newcommand{\thetitle}{Геометрические модели дислокаций в кристаллах (реферат по ин.яз.)}
  \newcommand{\theauthor}{Иван Новиков}

  \author{\theauthor,\\ кафедра физики и информационных систем КубГУ}
  \title{\thetitle}
  \date{\today{} \currenttime}

  \SetPdfTitleAndAuthor{\thetitle}{\theauthor}

\begin{document}

%----------------------- титульный лист ------------------------

  \maketitle
  \newpage

%------------------------- содержание -------------------------

  \microtypesetup{protrusion=false} % отключить protrusion для содержания
    \clearpage
    \tableofcontents
  \microtypesetup{protrusion=true}

%-------------------------- введение --------------------------

  \sectiontoc{Введение}\label{sec:intro}

  Полупроводники в настоящее время имеют огромное значение, поскольку лежат в основе всей
  современной электроники и компьютерной техники, а также применяются в солнечной энергетике и
  оптоэлектронике. При этом изучение структурных дефектов в полупроводниках представляет особый
  интерес, с~одной стороны, поскольку они могут существенно влиять на электрические свойства
  полупроводниковых приборов (как отрицательно, так и положительно), а с другой~--- по причине их
  участия в механических процессах, таких как пластичность, причём последнее имеет место в широком
  смысле: не только для полупроводников, но и для других кристаллических веществ.

  Среди различных типов дефектов кристаллической решётки особенный интерес представляют линейные,
  или одномерные дефекты~--- дислокации, оказывающие наибольшее влияние на электрические и
  механические свойства кристалла. Это такие дефекты, которые искажают правильное расположение
  атомных (кристаллографических) плоскостей. Два базовых типа: краевая и винтовая дислокации.
  \term{Краевая дислокация} соответствует атомной плоскости, обрывающейся внутри кристалла.
  \term{Винтовую дислокацию} можно представить себе как результат сдвига на период решётки одной
  части кристалла относительно другой вдоль некоторой полуплоскости параллельно её краю, играющему
  роль оси дислокации. % --- Физическая энциклопедия (http://www.femto.com.ua/articles/part_1/1036.html)

  Помимо существенного влияния на свойства кристалла, есть и другая причина, по которой
  дислокации представляют интерес для учёных. Несмотря на то, что данное явление широко изучено
  экспериментально, до сих пор не существует единой общепринятой теории, описывающей возникновение и
  динамику дислокаций и хорошо согласующейся с экспериментом.

  В данной работе рассмотрены несколько последних разработок в области теории дислокаций,
  опирающихся на разные подходы, но имеющих общую цель: создание полной и пригодной для
  компьютерного моделирования теории, согласующейся с экспериментом и позволяющей описывать
  связанные с дислокациями явления, такие как пластичность.

  \section{Теория пластичности и динамика дислокаций на основе механики сплошных сред}

  В работе~\cite{hochrainer-cdd} предлагается строить теорию динамики деформаций и пластичности на
  основе классических уравнений динамики деформируемых сплошных сред.

  Многие явления, связанные с пластичностью металлов (такие как наклёп), могут быть напрямую
  объяснены за счёт взаимодействия и размножения дислокаций. Тем не менее, как отмечают авторы
  данной работы, большинство существующих теорий не охватывают перемещения дислокаций в полной мере.
  Дело в том, что для традиционных инженерных приложений эта проблема не является существенной.
  Более того, обычным подходом для таких приложений вообще является отказ от рассмотрения дислокаций
  в пользу простой параметризации, подгоняемой под результаты механических испытаний.

  % In small scale structures, however, local constitutive laws have been challenged...


  \subsection{Непрерывная теория дислокаций} % TODO Как лучше перевести Continuum Dislocation Dynamics (CDD)?
  \subsection{Результаты моделирования}

  \section{Полевая теория многомасштабной пластичности}

  В работе~\cite{hasebe-ftmp} предпринимается попытка объединить наработки калибровочной,
  дифференциально-геометрической и квантовой теорий поля в обобщённую теорию~--- так называемую
  полевую теорию многомасштабной пластичности.

  \section{Калибровочная теория дислокаций в двумерных нематиках}

  В работе~\cite{liu-nematic} \dots

  \sectiontoc{Заключение}

  Таким образом, \dots

  \PrintBibliography

\end{document}
