\documentclass[a4paper, 14pt, titlepage]{extarticle}
  \usepackage[bib,math,figtable,titlegost,newpage]{mypreamble} % моя преамбула
  \usepackage[babel,protrusion=true,expansion]{microtype}
  \usepackage{setspace} % для \begin{spacing}

  \newcommand{\thetitle}{Современные геометрические модели дислокаций и пластичности кристаллов}
  \newcommand{\theauthor}{Иван Новиков}

  \author{\theauthor,\\ кафедра физики и информационных систем КубГУ,\\ направление аспирантуры 01.04.10}
  \title{\thetitle}
  \date{\today{} \currenttime}

  \SetPdfTitleAndAuthor{\thetitle}{\theauthor}

  % некоторые распространённые индексы в формулах:
  \newcommand{\tot}{\textup{t}}    % - полная (плотность)
  \newcommand{\plast}{\textup{pl}} % - пластическая (деформация)
  \newcommand{\elast}{\textup{e}}% - упругая (энергия)
  \newcommand{\II}{\textup{II}}    % - второго порядка (тензор \alpha)
  \newcommand{\inc}{\textup{inc}}  % - несовместный
  \newcommand{\inh}{\textup{inh}}  % - неоднородный
  \newcommand{\refer}{\textup{ref}}% - опорный (reference)
  \newcommand{\sys}{\textup{sys}}  % - системы
  % некоторые распространённые тензоры и объекты
  \newcommand{\lfk}{\vect{l}_{\phi_{\vect{\kappa}}}}
  \newcommand{\R}{\mathbb{R}}  % - пространство R
  \newcommand{\Sklj}{S^{..j}_{kl.}}  % - тензор кручения
  \newcommand{\Rklmn}{R^{...n}_{klm.}} % - тензор кривизны

\begin{document}

%----------------------- титульный лист ------------------------

  \thispagestyle{empty}
  \begin{center}
  \begin{spacing}{1.0}
  {\small МИНИСТЕРСТВО ОБРАЗОВАНИЯ И НАУКИ РОССИЙСКОЙ ФЕДЕРАЦИИ\\
  \textit{Федеральное государственное бюджетное образовательное учреждение\\
  высшего профессионального образования}}\\
  \textbf{\large«Кубанский государственный университет»\\
  (ФГБОУ ВПО «КубГУ»)}

  \vspace {5mm}

  Кафедра физики и информационных систем

  \vspace {4cm}

  \textbf{\large Р~Е~Ф~Е~Р~А~Т}\\
  по дисциплине <<Иностранный язык в специальности>>

  \vspace {0.5cm}

  { \scshape \thetitle }

  \vspace {1.5cm}

  \hfill\begin{minipage}{0.63\textwidth}
    Выполнил:\\
    аспирант кафедры физики и информационных систем
    физико-технического факультета\\
    НОВИКОВ Иван Александрович
    %Направление аспирантуры 01.04.10 Физика полупроводников
    \vspace{1cm}

    Научный руководитель:\\
    д-р физ.-мат. наук, профессор,\\
    Богатов Н.\,М.
  \end{minipage}

  \vfill

  Краснодар \the\year
  \end{spacing}
  \end{center}

%------------------------- содержание -------------------------

  \microtypesetup{protrusion=false} % отключить protrusion для содержания
    \clearpage
    \tableofcontents
  \microtypesetup{protrusion=true}

%-------------------------- введение --------------------------

  \sectiontoc{Введение}\label{sec:intro}

  Полупроводники в настоящее время имеют огромное значение, поскольку лежат в основе всей
  современной электроники и компьютерной техники, а также применяются в солнечной энергетике и
  оптоэлектронике. При этом изучение структурных дефектов в полупроводниках представляет особый
  интерес, с~одной стороны, поскольку они могут существенно влиять на электрические свойства
  полупроводниковых приборов (как отрицательно, так и положительно), а с другой~--- по причине их
  участия в механических процессах, таких как пластичность, причём последнее имеет место в широком
  смысле: не только для полупроводников, но и для других кристаллических веществ.

  Среди различных типов дефектов кристаллической решётки особенный интерес представляют
  \term{дислокации}~--- линейные (одномерные) дефекты, оказывающие наибольшее влияние на электрические и
  механические свойства кристалла. Это такие дефекты, которые искажают правильное расположение
  атомных (кристаллографических) плоскостей. Два базовых типа: краевая и винтовая дислокации.
  \term{Краевая дислокация} соответствует атомной плоскости, обрывающейся внутри кристалла.
  \term{Винтовую дислокацию} можно представить себе как результат сдвига одной
  части кристалла относительно другой вдоль некоторой полуплоскости параллельно её краю,
  в~результате чего атомные плоскости винтообразно изгибаются.

  Помимо существенного влияния на свойства кристалла, есть и другая причина, по которой
  дислокации представляют интерес для учёных. Несмотря на то, что данное явление широко изучено
  экспериментально, до сих пор не существует единой общепринятой теории, количественно описывающей
  возникновение и динамику дислокаций и хорошо согласующейся с экспериментом.

  В реферате рассмотрены несколько последних разработок в области теории дислокаций,
  опирающихся на разные подходы, но имеющих общую цель: создание полной и пригодной для
  компьютерного моделирования теории, согласующейся с экспериментом и позволяющей описывать
  связанные с дислокациями явления, такие как пластичность.

  \section{Теория пластичности и динамики дислокаций на основе механики сплошных сред}

  В работе~\cite{hochrainer-cdd} предлагается строить непрерывную теорию динамики деформаций и
  пластичности на основе классических уравнений динамики деформируемых сплошных сред.

  Многие явления, связанные с пластичностью металлов (такие как наклёп), могут быть напрямую
  объяснены за счёт взаимодействия и размножения дислокаций. Тем не менее, как отмечают авторы
  работы~\cite{hochrainer-cdd}, большинство существующих теорий не охватывают перемещения дислокаций в полной мере.
  Дело в том, что для традиционных инженерных приложений эта проблема не является существенной.
  Более того, обычным подходом для таких приложений вообще является отказ от рассмотрения дислокаций
  в пользу простой параметризации материальных уравнений, подгоняемой под результаты механических
  испытаний. Однако, при переходе к структурам малого масштаба локальные материальные уравнения
  перестают работать, так как не могут объяснить наблюдаемую зависимость от размеров.

  Вышесказанное объясняет потребность в непрерывной статистической теории дислокаций и пластичности. При
  формулировании же такой теории возникает фундаментальная сложность, связанная с выражением
  расположения и движения дислокаций в виде непрерывных величин. Авторы утверждают, что в своих
  прежних работах им удалось справиться с ней путём рассмотрения абстрактного
  конфигурационного пространства более высокого измерения, включающего дополнительно координаты,
  характеризующие направление дислокационной линии. Но получаемая таким образом теория является либо
  неточной (при малой размерности пространства), либо непригодной для эффективного компьютерного
  моделирования (при высокой размерности). Поэтому в работе~\cite{hochrainer-cdd} предпринимается попытка
  преодолеть эти трудности систематически, путём мультипольного разложения функции плотности.
  Получаемые при этом уравнения ложатся в основу теории поля, которой авторы дают название
  \term{непрерывная динамика дислокаций (НДД)}.

  \subsection{Непрерывная динамика дислокаций} % TODO можно ли лучше перевести Continuum Dislocation Dynamics (CDD)?

  Необратимая (пластическая) деформация возникает в результате \term{скольжения дислокаций}. Для описания
  этого процесса традиционно используются следующие термины и характеризующие их численные величины. Скольжение
  происходит вдоль определённых плоскостей, называемых \term{плоскостями скольжения} и
  характеризуемых нормалью $\vect{n}$, и в определённых направлениях, называемых \term{направлениями
  скольжения}, задаваемых вектором $\vect{m}$. Дислокации характеризуются \term{вектором Бюргерса}
  $\vect{b} = b \vect{m}$, выражающим сдвиг в плоскости скольжения, и направлением дислокационной
  линии $\vect{l}$.  Плотность дислокаций задаётся двумя классическими величинами: длиной
  дислокационной линии в единице объёма $\rho_\tot$~--- \term{полной плотностью дислокаций} и
  \term{тензором плотности дислокаций} $\matx{\alpha}$. % определения через $\matx{\beta}^\plast$ и $\vect{b}$ опущены
  Авторы работы~\cite{hochrainer-cdd} предлагают строить свою теорию на основе аналога этого тензора
  в вышеупомянутом пространстве большей размерности~--- \term{тензора плотности дислокаций второго
  порядка} $\matx{\alpha}^\II$, получаемого за счёт добавления угла $\phi$ между векторами $\vect{l}$ и
  $\vect{b}$ в качестве независимого параметра.

  Для вывода новых уравнений и построения тензора $\matx{\alpha}^\II$ авторы выбирают определённую
  систему координат (в которой $\vect{b} = (b,0,0)$) и рассматривают две функции плотности.
  Первая, $\rho(p,\phi)$, выражает количество дислокаций, проходящих через единичный элемент площади в
  точке $p$ в направлении $\vect{l}(\phi)$. Вторая, \term{плотность кривизны} $q(p,\phi)$,
  характеризует изменение направления вдоль дислокационной линии и связана с локальной кривизной
  дислокационной линии $k$ следующим образом: $q(p,\phi) = \rho(p, \phi) k(p, \phi)$.
  Чтобы преодолеть главную проблему такого подхода~--- повышенные вычислительные затраты из-за
  увеличенной размерности~--- авторы применяют разложение этих функций в ряд Фурье и используют
  коэффициенты этого разложения в качестве переменных упрощённой теории НДД.
  Первые из этих коэффициентов представляют собой хорошо известные и упомянутые ранее величины: это
  полные плотности дислокаций ($\rho_\tot$) и кривизны ($q_\tot$), вектор плотности дислокаций
  $\vect{\kappa}(p) = (\kappa^1,\kappa^2,0)$ и классический тензор плотности дислокаций $\matx{\alpha}$.

  При построении эволюционных уравнений для усреднённых величин возникает бесконечная цепочка
  зависимостей: коэффициенты Фурье более низких порядков выражаются через коэффициенты более высоких
  порядков. Чтобы получить конечный набор уравнений, требуется аппроксимация путём замыкания на
  определённом порядке. В своей работе авторы стремятся замкнуть систему на наименьшем возможном
  порядке, то есть получить описание динамики дислокаций с наименьшим количеством свободных
  переменных. По их мнению, для материалов с изотропной подвижностью дислокаций такой минимальный
  набор~--- это величины $\rho_\tot$, $q_\tot$ и $\vect{\kappa}$. В результате выкладок они приходят
  к уравнению
  \[
    \partial_t \rho_\tot = - \nabla \cdot \left( \nu \vec{\kappa}^\bot \right) + \nu q_\tot ,
  \]
  где $\vect{\kappa}^\bot = (\kappa^2, -\kappa^1, 0)$, то есть $\vect{\kappa}^\bot \bot \vect{\kappa}$,
  а $\nu$~--- скорость дислокации.

  Это уравнение и есть основной результат предлагаемой авторами теории. Оно может быть интерпретировано
  следующим образом: поскольку дислокации с взаимно противоположными $\vect{l}$ движутся во
  встречных направлениях, то к изменению полной длины дислокационной линии приводит только разница
  между этими противонаправленными потоками. Это эволюционное уравнение для $\rho_\tot$ дополнено
  эволюционным уравнением для $\vect{\kappa}$, которое выводится исходя из того, что
  $\matx{\alpha} = \nabla \times \matx{\beta}^\plast$ и $\matx{\alpha} = \vect{\kappa} \otimes \vect{b}$:
  \[
    \partial_t \vect{\kappa} = \nabla \times (\rho_\tot vn).
  \]
  Наконец, система уравнений завершается эволюционным уравнением для $q_\tot$. Выведя его, авторы
  отмечают, что его более удобно записать в другой форме, используя $\bar{k}$ вместо $q_\tot$:
  \[
    \partial_t \bar{k} = -\nu \bar{k}^2 - \left(
      \frac{\rho_\tot+\kappa}{2\rho_\tot} \nabla^2_{\lfk,\lfk}\nu +
      \frac{\rho_\tot-\kappa}{2\rho_\tot} \nabla^2_{\lfk^\bot,\lfk^\bot}\nu
    \right) + \frac{\kappa}{\rho_\tot} \left(
      \bar{k}\nabla_{\lfk^\bot}\nu -
          \nu\nabla_{\lfk^\bot}\bar{k}
    \right).
  \]
  Здесь, вдобавок в вышеупомянутым, введены следующие обозначения: $\lfk = \vect{\kappa}/\kappa$~---
  среднее направление линии геометрически необходимых дислокаций, $\lfk^\bot = \vect{\kappa}^\bot/\kappa$
  --- ортогональный ему вектор, $\nabla^2_{\vect{x},\vect{x}} \equiv \nabla_{\vect{x}} \nabla_{\vect{x}} - \nabla_{\nabla_{\vect{x}}\vect{x}}$
  --- производная второго порядка. Физический смысл уравнения заключается в следующем. Первое слагаемое
  описывает эволюцию кривизны, полученную от расширения (со скоростью $\nu$) окружностей радиуса $r = 1/\bar{k}$.
  Слагаемое, содержащее вторые производные, описывает эволюцию кривизны в неоднородном поле скоростей.
  Предпоследнее слагаемое, содержащее $\nabla_{\lfk^\bot}\nu$, в случае $\rho_\tot = \kappa$ даёт
  вторую производную скорости вдоль линии дислокации (с обратным знаком):
  $-\nabla_{\vect{l}}\nabla_{\vect{l}}\,\nu$.
  Последнее слагаемое объясняет изменение кривизны в результате протекания градиента кривизны через
  фиксированную точку в пространстве.

  Система из трёх последних эволюционных уравнений представляет собой очень сжатое описание развития
  искривлённых дислокаций. Авторы показывают на примерах, что эта система действительно
  определяет непрерывную теорию динамики дислокаций.

  \subsection{Результаты моделирования}

  Первый пример показывает, что упрощённые уравнения сжатой теории корректно предсказывают потоки
  дислокаций и, тем самым, кинематически корректно фиксируют эволюцию нетривиальных дислокационных
  конфигураций. Рассматривается эволюция системы дислокаций в квадратичной области со стороной в
  20~мкм, окружённой непроницаемыми границами. Изначальные дислокации создаются в меньшей квадратной
  области в центре. Для простоты дислокации считаются невзаимодействующими. В ходе моделирования
  дислокации сначала распространяются без взаимодействия с границами, создавая область с повышенной
  плотностью дислокаций в центре ячейки. Затем они начинают накапливаться у границ и это накопление
  распространяется в сторону углов ячейки (рисунок~\ref{fig:hochrainer-cdd}).

  \includefigure[width=\textwidth]{Моделирование непрерывной динамики дислокаций: первый пример в момент времени $t=9.0$~мкс~\cite{hochrainer-cdd}}{hochrainer-cdd}

  Авторы отмечают, что результаты их модели, в отличие от
  феноменологических теорий пластичности, показывают отсутствие заметной связи между распределениями
  пластического скольжения и плотности дислокаций. Приводится также сравнение сжатой теории НДД с
  теорией высокой размерности, считающейся точной. Сравнение показывает, что рассчитанные значения
  плотности дислокаций и пластического скольжения не отличаются, и только для кривизны наблюдается
  небольшое различие между этими теориями. Зато время вычислений может быть существенно уменьшено
  за счёт применения упрощённой теории: с 9 часов до 5 минут.

  В другом примере рассматривается явление так называемого <<механического отжига>>, которое
  существующие непрерывные теории пока не способны воспроизвести. Оно заключается в том, что после
  приложения сжимающей нагрузки плотность дислокаций резко снижается. При этом на диаграмме
  деформирования после некоторого отрезка пластической деформации вновь восстанавливается упругое
  поведение. Авторы описывают, от каких особенностей и условий реального эксперимента пришлось
  абстрагироваться для упрощения задачи. Приведённые результаты моделирования качественно
  воспроизводят восстановление упругости после режима пластичности, наблюдаемое в эксперименте, в то
  время как количественно воспроизвести его невозможно из-за вышеупомянутых упрощений условий и
  стохастического характера самого эксперимента.

  Авторы делают вывод, что сжатое описание, предлагаемое ими, является важным шагом на пути к
  физически обоснованной непрерывной теории пластичности. Наиболее важные эффекты, связанные с
  дислокациями, могут быть успешно описаны с помощью предложенных уравнений. Разумеется, кроме
  этих кинематических уравнений в обобщённой теории дислокаций и пластичности должны
  присутствовать ещё и взаимодействие между дислокациями, движение вне плоскости и анизотропная
  подвижность дислокаций, но главное, чтобы в её основе была полная кинематическая модель, такая как
  предложенная авторами.

  \section{Полевая теория многомасштабной пластичности}

  В работе~\cite{hasebe-ftmp} предпринимается попытка объединить опыт применения калибровочной,
  дифференциально-геометрической и квантовой теорий поля для моделирования дислокаций в обобщённую
  теорию~--- так называемую \term{полевую теорию многомасштабной пластичности} (ПТМП). Её целью
  является воспроизведение неоднородных дислокационных структур, экспериментально наблюдаемых с
  помощью просвечивающей электронной микроскопии и дифракции отражённых электронов.

  Многие экспериментальные наблюдения подтверждают чрезвычайную важность дислокационных структур,
  возникающих при упруго"=пластической деформации металлических кристаллов и контролирующих
  механические свойства этих материалов. Было, соответственно, предпринято огромное количество
  попыток создания многомасштабных подходов к этим сложным явлениям. Однако в них не придаётся
  большого значения эволюции таких структур. Вероятно, поэтому до сих пор не существует пригодных
  для практического использования многомасштбаных схем моделирования.

  Как утверждают авторы работы~\cite{hasebe-ftmp}, успехи полевой теории многомасштабной
  пластичности позволили им совместить эти схемы с передовыми технологиями измерения деформаций. Это
  привело их к созданию совершенно новой техники для явного расчёта неоднородностей, развивающихся
  при деформациях кристаллических материалов. В этом подходе измеренная упругая деформация
  обрабатывается путём расчёта тензора несовместности и флуктуационной составляющей энергии
  деформации с тем, чтобы далее рассмотреть развитие неоднородностей внутри и между зёрнами с точки
  зрения ПТМП.

  \subsection{Основные теоретические идеи}

  Важной особенностью ПТМП является её способность напрямую работать с вызванными деформацией полями
  неоднородности с точки зрения
  \begin{enumasbuk}
    \item \term{эволюции}: как и почему эти неоднородности возникают и развиваются; \label{enu:evol}
    \item \term{описания}: как они могут быть математически выражены; \label{enu:desc}
    \item \term{взаимодействия}: как и почему они взаимодействуют друг с другом. \label{enu:coop}
  \end{enumasbuk}

  \term{Дифференциально-геометрическая теория поля} (ДГТП), базирующаяся на теоретических основах
  \term{неримановой пластичности} отвечает за \ref{enu:desc} описание. Два тензора~--- кручения и
  кривизны --- дают полное геометрическое описание неоднородности любого масштаба. Более того, можно
  построить на основе этой теории формализм \ref{enu:coop} взаимодействия между разными масштабами
  с~геометрической точки зрения.

  Одна из движущих сил эволюции полей неоднородностей~--- это <<коллективные>> эффекты, вызываемые
  чрезвычайно большим количеством взаимодействующих дислокаций, приводящих к развитию ячеистых
  структур. Эти аспекты могут быть рассмотрены с помощью \term{квантовой теории поля} (КвТП), тем
  самым получая \ref{enu:evol} эволюцию. При этом отправной точкой для применения КвТП является
  \term{калибровочная теория поля} (КлТП), поскольку она сообщает нам, как однозначно записать
  гамильтониан системы в соответствии с калибровочно"=инвариантным лагранжианом. Более того, КлТП
  даёт математически обоснованную базу для динамических аспектов ДГТП с точки зрения того, как
  энергия упругой деформации должна быть преобразована в степени свободы дефектов (дислокаций), т.е. \ref{enu:desc}.

  Предлагаемая теория ПТМП базируется на ДГТП и путём добавления дополнительной гипотезы, описанной
  далее, объединяет в себе также важнейшие особенности КвТП и КлТП. Кратко этот базис ДГТП
  заключается в следующем. Ковариантная производная пространства определяется через коэффициенты
  связности $\Gamma^j_{kl}$ как
  \[
    \nabla x^i = \dif x^i + \Gamma^j_{kl} x^l \dif x ^k,
  \]
  а тензоры кручения и кривизны задаются как
  \begin{align*}
    \Sklj  & = \Gamma^j_{[kl]}, \\
    \Rklmn & = 2\left( \partial_{[k}\Gamma^n_{l]m} + \Gamma^n_{[k|p|}\Gamma^p_{l]m} \right).
  \end{align*}
  Если тензор кручения $\Sklj$ отличен от нуля, то коэффициенты связности $\Gamma^j_{kl}$ описывают
  нериманово пространство. Свёртывание этих тензоров с учётом симметрии приводит к получению хорошо
  известных тензоров второго ранга, называемых (соответственно) \term{тензором плотности дислокаций}
  $\alpha_{ij}$ и \term{тензором несовместности} $\eta_{ij}$:
  \begin{align*}
    \alpha_{ij} = - \epsilon_{ikl} \partial_k \beta^{\plast}_{lj}
              & = \frac{1}{2} \epsilon_{ikl}\Sklj,\\
    \eta_{ij}   = \epsilon_{ikl}\epsilon_{jmn} \partial_k\partial_m \varepsilon^{\plast}_{ln}
              & = \frac{1}{4g}\epsilon_{ikl}\epsilon_{jmn}\Rklmn,
  \end{align*}
  где $\beta^{\plast}_{lj}$ и $\varepsilon^{\plast}_{ln}$~--- тензоры пластического искривления и
  пластической деформации.

  Предлагаемая гипотеза, называемая авторами \term{<<текуче-эволюционной>>} предлагает обобщённый
  закон развития полей неоднородностей и пластического течения, сопровождающегося диссипацией
  энергии. В этом законе может быть воплощено понятие <<дуализма>> между гидростатическим
  напряжением и полями деформации, вводимое в ПТМП. Впрочем, этот закон пока остаётся гипотезой,
  требующей подтверждения. Её краткое обоснование заключается в следующем.

  На основе того факта, что и тензор энергии-импульса $T_{ij}$, и тензор несовместности $\eta_{ij}$
  имеют нулевую дивергенцию, можно определить соответствующие им сохраняющиеся векторные величины:
  \[
    \left\{
      \begin{aligned}[c]
        \partial_b \eta_{ab} &= 0\\
        \partial_b \delta T_{ab} &= 0
      \end{aligned}
    \right.
    \quad \Leftrightarrow \quad
    \left\{
      \begin{aligned}[c]
        \delta u^\inc_a &\equiv u^\refer \int_S n_b \eta_{ab} \dif S \\
        \delta f^\inh_a &\equiv \int_S n_b \delta T_{ab} \dif S
      \end{aligned}
    \right. ,
  \]
  названные авторами, соответственно, \term{<<несовместным>> смещением} и \term{<<неоднородной>>
  силой}, где индексы $a,b = 1,2,3,4$ ($x_4$~--- это время $t$). Следует отметить, что вместо самого
  тензора энергии-импульса $T_{ab}$ здесь используется только его флуктуационная часть
  $\delta T_{ab} \equiv T_{ab} - \langle T_{ab} \rangle_\sys$, где $\langle \cdot \rangle_\sys$
  обозначает пространственно"=временное среднее по всей системе.

  Предполагая, что рост <<несовместного>> смещения вызывается <<неоднородной>> силой, авторы
  постулируют линейную зависимость между ними и получают
  \[
    \eta_{ab} = \kappa_{abcd} \delta T_{ab}.
  \]

  С учётом пространственно-временной изотропии, после рассмотрения временн\'{о}й компоненты и
  предположения статических условий выражение сводится к
  \[
    \eta_{AA} = \kappa \delta \mathcal{U}^\elast,
  \]
  где $A = 1,2,3$.

  Для применения вышеописанных понятий теории поля к практическим применениям, авторы предлагают
  материальное уравнение, выражающее зависимость между касательными
  напряжением и деформацией, а затем подробно рассматривают отдельные входящие в него члены:
  напряжение торможения и остаточное микронапряжение, соотношение упрочнения и члены, соответствующие
  градиенту деформации.

  Авторы описывают условия двух случаев моделирования, проводимого ими на основе этих уравнений,
  соответствующих двум видам экспериментальных наблюдений.
  В первом случае рассматривается монокристаллическая сталь, подвергаемая сдвигу в четырёх
  кристаллографических ориентациях, а развивающиеся дислокационные структуры наблюдаются с помощью
  просвечивающей электронной микроскопии (ПЭМ). В другом~--- поликристаллический образец, деформируемый
  натяжением, а деформация между зёрнами измеряется с помощью технологии дифракции отражённых электронов (ДОЭ).
  Моделирование в обоих случаях производится с помощью метода конечных элементов.

  \subsection{Результаты моделирования}

  Для первого вида эксперимента наблюдаемые с помощью ПЭМ микроструктуры можно разделить на три
  типа: состоящие из одного набора параллельных дислокационных границ, из двух таких наборов и из
  равноосных ячеек. Микрофотографии сравниваются с результатами моделирования, которые успешно
  воспроизводят эти виды дислокационных структур (рисунок~\ref{fig:hasebe-ftmp-tem}).

  \begin{myfigure}{Моделирование дислокационных структур с помощью ПТМП~\cite{hasebe-ftmp}: сравнение с экспериментальным наблюдением}{fig:hasebe-ftmp}
    \subfigure[width=0.45\textwidth]{Монокристаллическая сталь: структуры на ПЭМ"=микрофотографии}{hasebe-ftmp-tem}
    \hspace{0.05\textwidth}
    \subfigure[width=0.45\textwidth]{Поликристалл: контуры деформации в ДОЭ"=изображении}{hasebe-ftmp-ebsd}
  \end{myfigure}

  Во втором случае авторы сравнивают экспериментально полученные и смоделированные контуры
  деформации (рисунок~\ref{fig:hasebe-ftmp-ebsd}). Особое внимание они обращают на зёрна, имеющие
  жёсткость больше или меньше, чем предполагается. Кроме этой особенности моделирование успешно
  воспроизводит неровности, возникающие внутри зёрен. Из анализа диаграммы зависимости $\tr\eta$ от
  $\mathcal{U}^\elast$ авторы делают вывод, что структуры дислокаций ведут себя как накопители
  энергии: сохраняют излишнюю энергию деформации, которая может быть поглощена впоследствии.

  Авторы подводят итог, подчёркивая эффективность применения тензора несовместности благодаря его
  способности описывать микроскопические степени свободы и определять поток энергии из упругих
  состояний в пластические. Измерение и моделирование дислокационных структур является, по их
  мнению, критичным для понимания упруго"=пластических свойств материала. Предлагаемая же ими схема
  может быть легко расширена до трёхмерного случая. При этом, разумеется, многие детали, касающиеся
  механизмов влияния этих структур на многомасштабные реакции, заслуживают дальнейшего изучения.

  \section{Калибровочная теория дислокаций в 2D"=нематиках}

  Одно из самых экзотических состояний материи~--- это \term{нематическая фаза} жидких кристаллов.
  В~этом состоянии вещество обладает как свойствами жидкости (способность течь), так и свойствами
  кристаллов (упорядоченность молекул, анизотропия). Фазовые переходы между нормальным и
  нематическим состоянием происходят при участии дислокаций. В работе~\cite{liu-nematic}
  рассматривается классификация нематических фаз в двумерных кристаллах и предлагается
  $O(2)/Z_N$"=калибровочная теория, описывающая такие фазовые переходы.

  В~физике вообще, а особенно в физике кристаллов, большое внимание уделяется \term{симметрии}
  системы. В кристаллах особенно ярко проявляются \term{трансляционная} симметрия, то есть
  инвариантность относительно сдвига, кратного заданному вектору, и \term{осевая} симметрия~---
  инвариантность относительно поворота на угол, кратный $360^\circ/n$.

  Авторы опираются на понятие \term{спонтанного нарушения симметрии}, описывающее переход физической
  системы из симметричного, но беспорядочного состояния в упорядоченное состояние, в котором
  исходная симметрия системы утеряна. Это имеет место и в случае перехода в нематическую фазу: если
  в кристаллическом состоянии нарушены и трансляционная, и осевая симметрия, то в результате
  конденсации дислокаций трансляционная симметрия восстанавливается, в то время как осевая остаётся
  нарушенной. При этом получаются различные подгруппы осевой симметрии, происходящие из различных
  пространственных групп, лежащих в основе исходного кристалла.

  При рассмотрении нарушенной осевой симметрии и получающейся в результате классификации нематиков
  крайне важна размерность пространства, поскольку двумерная ортогональная группа $O(2)$ является
  \term{абелевой}, в то время как группа трёхмерных вращений $O(3)$ неабелева. Поэтому авторы
  работы~\cite{liu-nematic} решают рассмотреть классификацию нематических фаз в двумерном абелевом
  случае и попытаться выделить для дальнейшего исследования общие принципы, которые могут быть
  применимы и к неабелеву трёхмерному случаю.

  Традиционно слово \term{нематик} относится к фазам, в которых осевая симметрия нарушается
  столбчатыми молекулами, имеющими симметрию $C_2$, а для фаз с симметрией $C_6$ был придуман термин
  \term{гексатик}. Однако может существовать множество фаз с различным образом нарушенной осевой
  симметрией, которые не имеет смысла обозначать отдельно, тем более, что все они нарушают симметрию
  одинаковым образом и, следовательно, могут рассматриваться как \term{нематики}. Поэтому авторы
  употребляют следующую номенклатуру: фаза с остаточной осевой симметрией $H$, где $H$~--- подгруппа
  $O(2)$, называется $H$-нематиком. Они показывают, что конденсация дислокаций приводит
  к~возникновению пяти различных классов нематических фаз, инвариантных относительно различных
  дискретных подгрупп $O(2)$~--- групп $C_N$ при $N = 1,2,3,4,6$. Получаемые нематики называются,
  соответственно, $C_N$-нематиками.  Обобщая трёхмерную калибровочную теорию $Z_2$, описывающую
  одноосные нематики с симметрией $D_{\infty h}$, они строят общую $O(2)/Z_N$"=калибровочную теорию
  решётки для всех $C_N$ фаз путём связывания материальных полей $O(2)$ с~калибровочной теорией
  решётки $Z_N$.

  Прежде всего, очевидно, что конденсация дислокаций приводит к нематической фазе, нарушающей только
  осевую симметрию, поскольку вектор Бюргерса привязан к решётке Браве. Кроме того, дислокации имеют
  внутреннюю симметрию, налагаемую \term{пространственной группой}, составляя дефект, отвечающий
  только трансляционной симметрии. Процесс размножения дислокаций восстанавливает трансляционную
  симметрию и потому описывает фазовый переход между кристаллическим и нематическим состоянием.
  В~терминах пространственных групп этот переход может быть описан соответствующей группой симметрии.
  Авторы начинают с рассмотрения евклидовой группы $E(2)$, элементы которой $(\matx{A},\vect{t})$
  выполняют поворот $\matx{A} \in O(2)$ и смещение $t \in \R^2$: $\vect{r} \mapsto \matx{A}\vect{r} + \vect{t}$.
  От $E(2)$ они переходят к её пространственной подгруппе $G$ такой, что её смещения
  $T = \{\vect{t} \mid (\matx{I},\vect{t}) \in G\}$ являются линейными комбинациями векторов
  примитивной ячейки $\vect{t}_i$. Важно отметить, что $G/T$ изоморфна точечной группе $P$.

  Вектор Бюргерса фиксирован и может равняться только линейной комбинации векторов элементарной
  ячейки. В отсутствие дисклинаций вектор Бюргерса каждой дислокации является сохраняющейся
  величиной и получающаяся фаза всё же имеет осевую симметрию, которая в точности описывается
  точечной группой оригинальной решётки Браве. Следовательно, можно обозначить фазы нематиков по их
  $C_N$ инварианту. В итоге, из точечных групп, лежащих в основе решёток Браве, немедленно
  получается, что фазы $C_2$, $C_4$ и $С_6$ могут быть получены путём дислокационно"=опосредованного плавления.

  Далее авторы описывают построение $O(2)/Z_N$"=калибровочной теории для всех этих фаз с двумя
  параметрами $J$ (нематическое взаимодействие) и $K$ (подавление дефектов), связанными с
  калибровочными полями. и рассматривают получаемую фазовую диаграмму, которая содержит три фазы:
  $C_N$"=нематик, изотропная жидкость и топологическая фаза. Отдельно они рассматривают
  предел сильной связи, соединяющий их теорию с традиционной теорией нематических фаз.

  \sectiontoc{Заключение}

  Таким образом, существует большое множество подходов к проблематике структурных дефектов в
  кристаллах, в частности, дислокаций. Предлагаются как статистические, так и калибровочные подходы.
  Некоторые термины и величины, такие как тензор плотности дислокаций, используются широко и
  присутствуют сразу во многих подходах. Развитие и объединение этих методов в общую теорию является
  перспективной задачей для исследователей, занимающихся изучением структурных дефектов.

  \PrintBibliography

\end{document}
